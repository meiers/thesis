\subsection{Traditional SV discovery}
\label{sec:sv_detection_old}

A traditional way to investigate the karyotype of cells is by arresting cells in
metaphase, staining chromosomes and observing them under a microscope
\citep{Speicher2005}. The images of each chromosome are then ordered by
chromosome to show an ideogram. By counting the number of each chromosome in the
cell, gross abnormalities such as aneuploidy can be detected, which is
schematically depicted under aneuploidy in \cref{fig:SV_classes}. Relevant
disease-linked forms of aneuploidy could be unraveled early on this way, e.g.
the trisomy of chromosome 21 \citep{Lejeune1959}. The classic technique has been
refined to allow higher specificity and resolution, mostly via improved ways of
staining, such as by quinacrine staining, Giemsa banding, or chromosome-specific
labeling based on in situ hybridization \citep{Speicher2005}. These techniques
principally allow the detection of \acp{cnv}, inversions and translocations,
yet the size range of these \acp{sv} has to be in the order of several Megabases
or larger. Fluorescence in situ hybridization \citep{Bauman1980}, which relies
on the annealing of fluorescently labeled DNA probes to their complement DNA,
is also applied in a targeted manner for validating predicted SVs of slightly
smaller size.

Other means of detecting \acp{sv} include optical mapping
\citep{Schwartz1993,Teague2010} and hybridization-based microarrays. The latter
ones, which are reviewed in \cite{Alkan2011}, used to be the dominating method
for \cnv detection before high-throughput sequencing became standard. One of the
two major techniques in this category, namely array comparative genomic
hybridization, utilizes the competition of the test sample's DNA and a reference
DNA to a hybridization probe (e.g. short oligonucleotides) to infer the relative
copy number of the tested locus \citep{Snijders2001}. Using high-density arrays,
this method can successfully detect deletions down to 500~bp in size
\citep{Conrad2010}. SNP arrays, on the other hand, utilize hybridization probes
at sites of polymorphic \acp{snv} to measure the allelic ratio within a single
sample, which is called \emph{\acf{baf}}. This way, \acp{cnv} but also \loh can
be detected.





\subsection{SV discovery in the era of massively parallel sequencing}
\label{sec:sv_detection_ngs}

Today, aforementioned techniques have largely been superseded by \sv detection
utilizing \acl{mps} data.
% although especially microarrays remain a cost-effective alternative.
\Ac{sv} detection methods based on \mps are often (also by \cite{Alkan2011})
separated into four different conceptual approaches, namely read pair analysis,
split-read analysis, read depth analysis, and sequence assembly. In practice,
\sv prediction tools do not necessarily fit into only one of these categories.
Below I will summarize the major ideas behind the different approaches as well
as representative software implementing them.

\paragraph{Paired-end analysis}
\Ac{sv} detection based on paired-end sequencing utilizes the orientation and
expected distance of two sequencing reads to another to detect rearrangements.
For instance, when their mapping distance on a reference genome is larger than
expected a deletion may have occurred in the test sample anywhere in between
those reads. Paired-end analysis can principally detect many different types of
\acp{sv}, including \acp{cnv}, inversions, translocations, insertions and
\acp{mei}. This technique had first been used on bacterial artificial chromosome
\citep{Volik2003}, on fosmid libraries \citep{Tuzun2005} and then on mate pair
sequencing of human genomes \citep{Korbel2007}. It is nowadays one of the
dominating principles of \sv detection and is implemented in well-established
software tools such as \textsc{BreakDancer} \citep{Chen2009}, \delly,
\textsc{CLEVER} \citep{Marschall2012}, or \textsc{LUMPY} \citep{Layer2014}.
Besides the richness in detectable SV classes, an advantage of the paired-end
read signature is that it can identify the breakpoints of an \sv. The breakpoint
accuracy depends on sequencing coverage and insert size distribution, but
it is typically in the range of few to several hundred base pairs.

\paragraph{Split-read analysis}
Split-read approaches utilize the fact that sequencing reads, if long enough,
can be divided and separately assigned to different locations of the
reference assembly. This is different from intra-read
gaps or mismatches, which are still tolerated within each alignment and
which are typically used to detect \acp{snv} and small indels.
Based on the position and orientation of the partial alignments, split-read
analysis detects \acp{sv} in the same way as paired-end analysis. The major
advantage, though, is that breakpoints can be determined much more accurately,
often down to the exact nucleotide.
Split-read approaches had been explored, for instance, early on during the
1000 Genomes Project (\cref{sec:1000G}) on 400~bp single-ended reads
\citep{Zhang2011}. Tools such as \textsc{Pindel} \citep{Ye2009} or
\textsc{BreakSeq} \citep{Lam2010} were among the first ones to specifically
implemented the split-read approach. Nowadays, strongly encouraged by increasing
read lengths form standard MPS machines (e.g. up to 2~x~300~bp on an Illumina
MiSeq platform nowadays), read mapping software has been refined towards the
ability to directly perform split-read mapping, as exemplified by tools such as
the widely used \bwamem or specialized tools like \textsc{YAHA}
\citep{Faust2012} and \textsc{SplazerS} \citep{Emde2012}.
This allowed other popular paired-end analysis detection tools to incorporate
the split-read approach, for example in \delly, \textsc{MATE-CLEVER}
\citep{Marschall2013}, and \textsc{LUMPY}.

\figuretextwidth[t]{SV_detection_methods_Tattini2015.jpg}{SV_detection}{Principles
    of SV detection in MPS data}{Schematic representation of the four distinct
    mechanisms for SV detection on three examples: (A) a deletion, (B) an
    insertion (of duplicated sequence, for example), and (C) an inversion.
    Figure modified from \citetitle{Tattini2015} \citep{Tattini2015} licensed
    under \acl{ccby4}.}

\paragraph{read depth analysis}
A complementary method for detecting \acp{cnv} utilizes the total read depth
signal inside an \sv. This resembles the methodology of micro-arrays, yet with
improved resolution since all of the (mappable) genome can be covered instead
of selected loci. \textsc{SegSeq} \citep{Chiang2009} and \textsc{CNV-seq}
\citep{Xie2009} were among the first tools that utilized read depth in a
sample-vs.-control scenario to detect \acp{cnv}. \textsc{mrFast}
\citep{Alkan2009_mrFAST} and \textsc{CopySeq} \citep{Waszak2010} extended this
approach to single-sample \cnv calling, and later \textsc{CNVnator}
\citep{Abyzov2011} and \textsc{genome STRiP} \citep{Handsaker2015} gained more
popularity. Normalization of read depth is a major challenge in these approaches
owing to an uneven sequencing coverage, which is why these tools typically
perform best when an internal reference (e.g. a control sample) is available.
The currently popular population-scale \cnv caller \textsc{genome STRiP} even
suggests a minimum of 20 to 30 sample genomes in order to perform well. In
contrast to paired-end or split-read detection, the read depth method cannot
accurately predict breakpoints, which becomes a major disadvantage especially
for smaller variants. For very large variants, on the other hand, read-depth
methods can still robustly predict \acp{cnv} even when their breakpoints reside
in repetitive regions.

\paragraph{Assembly} At last, sequence assembly-based methods do not rely on the
information provided by read mapping software but perform \textit{de novo}
assembly instead, as demonstrated for example by \citpe{Li2011a}.
While whole-genome assembly still remains computationally expensive
\citep{Bradnam2013}, there are methods that perform local re-assembly of reads,
for example \textsc{TIGRA} \citep{Chen2014b} or the recent \textsc{novoBreak}
\citep{Chong2017}. Especially for \sv types that are difficult to detect by
paired-end or split-read approaches, notably for insertion of novel DNA sequence,
assembly can yield a great benefit. Tools that address this are \textsc{NovelSeq}
\citep{Hajirasouliha2010}, \textsc{MindTheGap} \citep{Rizk2014}, and
\textsc{Basil/Anise} \citep{Holtgrewe2015}.


\subsection{State of the art and limitations of SV studies}
\label{sec:limitations}
\todo{write this section!}
