\cleardoublepage
\phantomsection
\chapter*{Summary}
\addcontentsline{toc}{chapter}{Summary}

\Acfp{sv} alter the structure of chromosomes by deleting, duplicating or
otherwise rearranging pieces of DNA. They contribute the majority of nucleotide
differences between humans and are known to play causal roles in many diseases.
Since the advance of \acf{mps} technologies, \acp{sv} have been
studied more comprehensively than ever before. However, in contrast to smaller
types of genetic variation, \sv detection is still fundamentally hampered by
the limitations of short-read sequencing that cannot sufficiently cope with the
complexity of large genomes. Emerging DNA sequencing technologies and protocols
hold the potential to overcome some of these limitations. In this dissertation,
I present three distinct studies each utilizing such emerging techniques to
detect, to validate and/or to characterize \acp{sv}. These
technologies, together with novel computational approaches that I developed, allow
to characterize \acp{sv} that had previously been challening, or even
impossible, to assess.

First, inversions---a class of \sv that is notoriously difficult to
ascertain---were studied in the context of the 1000 Genomes Project. These inversions had
previously been predicted from low-coverage short read sequencing data, but
remained inconclusive in classical \acs{pcr} validation experiments. Using sequencing
data from modern long-read technologies (\acl{pacbio} and \acl{ont}),
I was able to validate hundreds of them. I then developed a
computational tool to visualize long-read data, and discovered that the majority
of loci harbored complex \acp{sv} rather than simple inversions. These findings
suggest that the amount of complex structural variation in the human genome had
so far been under-appreciated, owing to limitations in their detection using
standard techniques.

In the second part, I explored the functional impact of large \acp{sv} on gene
expression and chromatin organization. Previously, a series of
studies described drastic effects of \acp{sv} on the regulation of genes
via mechanisms that alter the three-dimensional conformation of DNA. However,
these studies had focused on pathological phenotypes and on few,
selected genes. We hence set out to study
gene expression and chromatin conformation in highly rearranged
chromosomes of \textit{Drosophila melanogaster} without a pathological phenotype.
I first utilized \acs{hic}, which we applied in order to measure chromatin
conformation, to characterize the rearrangements present in these chromosomes.
Then, despite the presence of 15 breakpoints, we found no evidence for a
conformation-related mechanism acting on gene regulation. This is particularly
surprising as the majority of these breakpoints disrupted \aclp{tad}.
This study hence sheds a new light on the role of chromatin
conformation that is complementary to the findings of previous studies.
In addition, it demonstrates the capabilities of the \acs{hic} technology to
reveal structural variation.

% We applied \acs{hic} to measure the organization of chromatin into \aclp{tad}.
% I utilized this \acs{hic} data to determine the breakpoints of all 15 major
% rearrangements and other \acp{sv}, which had not been fully possible beforehand.
% Eventually, we observed a considerable amount of expression changes attributed
% to the presenece of structural variaiton, yet no evidence for the type of mechanism.
% These results first demonstrate the potential of \hic to facilitate \acp{sv}
% detection and, second, shed a new light on the role of chromatin conformation in
% gene regulation.

Third, I present the current state of a collaborative effort
to enable \sv detection in single cells. Studies of somatic mosaicism, i.e. on
the genetic heterogeneity among cells, have so far been severly limited in the
ability to discover \acp{sv}, especially copy-neutral and complex rearrangements.
We hence conceived a novel method to infer---for the first time---at least seven
different \sv classes in single cells. This approach utlizes three independent
signals that are identifiable in single-cell stranded template sequencing
(Strand-seq) data. I here
present a computational method called \mc to realize this idea and provide
examples that demonstrate its feasibility. In order to explore the limitations
of this method, I designed a versatile framework for the simulation of
Strand-seq data and used it to assess the performance of one of the central
steps of \mc. Once completed, this novel method will facilitate studies of \sv
heterogeneity and mosaicism in the context of cancer and ageing.

% Until now, the detection of \acp{sv} has been hampered by technological
% limitations. However, \sv detection is a fundamental first step towards
% studying the functional impact of this pervasive form of genetic variation.
% Here, I demonstrated how utilizing emerging sequencing technologies such as
% long-read sequencing, \hic, or Strand-seq overcomes many of these challenges.
% Based on novel computational approaches that I developed, I could characterize
% variants that had been difficult, if not impossible to ascertain beforehand.
% These tools were instrumental in gaining insights on the biology of \acp{sv} and
% they may, in the future, reveal more currently hidden patterns of structural
% variation. Together with efforts by others in the community these improvements
% in detection will allow us to gain a deeper understanding of the role of
% \acp{sv} in health and disease.

In summary, I utilized emerging technoogies to discover \acp{sv}---notably
copy-neutral and complex rearrangements---that so far eluded detection based on
\mps. This led to novel insights on the complexity and functional impact of these
\acp{sv}. Moreover, I developed computational tools that advance our capabilites
for \sv detection and characterization, and that might aid future studies to gain a
deeper understanding of the role of \acp{sv} in health and disease.




\cleardoublepage
\phantomsection
\chapter*{Zusammenfassung}%
\addcontentsline{toc}{chapter}{Zusammenfassung}%

Strukturvariationen (SV) verändern die Chromosomenstruktur, in dem sie
Teile der DNA deletieren, duplizieren oder anderweitig neu anordnen. Sie sind für
den Großteil der Unterschiede in der Nukleotidsequenz zwischen Menschen verantwortlich
und sind kausal für verschiedene Er\-kran\-kun\-gen. Seit dem Vormarsch
von hochparallelen Sequenziermethoden (\textit{massively parallel sequencing},
MPS) können SV umfassender denn je untersucht werden. Dennoch leidet die Detektion von
SV, im Gegensatz zu jener von kleineren Formen genetischer Variation, unter den
Limitierungen der Sequenzierung kurzer DNA Abschnitte, welche die Komplexität
großer Genome nicht ausreichend abbilden kann. Neu aufkommende
DNA-Se\-quen\-zier\-me\-tho\-den und –protokolle bergen das Potenzial diese Limitierungen
zu überwinden. In dieser Dissertation werden drei separate Studien präsentiert,
die aufkommende Technologien zur Detektion, Validierung und/oder Charakterisierung
von SV nutzen. Im Verbund mit neuartigen Computermethoden, die ich entwickelt habe,
erlauben diese Technologien die Bestimmung von SV die zuvor schwierig, wenn
nicht gar unmöglich, war.

Zunächst werden Inversionen---eine bekanntermaßen schwierig zu ermittelnde Form
von SV---im Rahmen des ``1000 Genomes Project'' untersucht. Diese Inversionen wurden zuvor
basierend auf MPS-Methoden mit niedriger Ab\-de\-ckung vorhergesagt, blieben aber
in klassischen \acs{pcr}-basierten Va\-li\-die\-rungs\-ex\-pe\-ri\-men\-ten ergebnislos.
Mithilfe moderner
Sequenziertechnologie für besonders lange DNA\--Ab\-schnit\-te von \acl{pacbio}
und \acl{ont} konnte ich hunderte davon validieren. Ich entwickelte dann eine
neue Methode zur Visualisierung dieser langen DNA-Abschnitte und fand heraus,
dass die Mehrzahl der Loci anstatt Inversionen letztendlich komplexe Formen von
SV enthielten. Diese Entdeckung legt nahe, dass die Menge an komplexen SV im
menschlichen Genom aufgrund der Limitierungen der verwendeten Technologien bisher
unterschätzt wurde.

% genauer cha\-rak\-te\-ri\-siert werden, welche zuvor in klassischen \acs{pcr}-basierten
% Validierungsexperimenten ergebnislos blieben. Es stellte sich heraus, dass die
% meis\-ten Loci mit vermuteten Inversionen tatsächlich komplexe Formen von SV
% enthalten, welche mit \mps Methoden und nur niedriger Abdeckung nicht zu
% erkennen sind. Diese Studie offenbarte eine ungeahnte Häufigkeit von komplexen
% Inversionen in der menschlichen Population und demonstriert die Nutzbarkeit
% jener Sequenziertechnologien zur Erkennung komplexer Formen von SV.

Im zweiten Teil untersuche ich den funktionellen Einfluss von großen SV
auf die Genexpression und die Chromatinorganisation.
Zuvor hatten eine Reihe von Studien drastische Effekte von SV auf die
Genexpression gezeigt, welche über einen Mechanismus funktionieren, der die
dreidimensionale Form der DNA be\-ein\-flusst. Diese Studien fokussierten sich
jedoch nur auf wenige Loci mit bekannten pathologischen Konsequenzen. Deshalb
entwickelten wir ein eigene Studie, in der wir die Genexpression und Veränderungen der
Chromatinstrukur in völlig neu angeordneten Chromosomen der Fruchtfliege
\textit{Drosophila melanogaster} ohne pathologischen Phänotyp untersuchen. Ich
nutzte zunächst die Hi-C Technologie, die wir zur Messung der Chromatinstruktur
verwendeten, um die neue Anordnung dieser Chromosomen zu bestimmen. Danach fanden
wir, obwohl 15 Bruchpunkte vorhanden waren, keine Hinweise auf einen funktionellen
Einfluss der Chromatinstruktur. Dies war besonders überraschend, da die Mehrzahl
dieser Bruchpunkte sogenannte \emph{\aclp{tad}} zu unterbrechen scheint. Unsere
Studie wirft somit ein völlig neues Licht auf die Rolle der Chromatinstruktur und
ergänzt damit vorherige Studien. Zusätzlich dazu wird in dieser Studie eindrucksvoll
demonstriert, wie SV mithilfe der Hi-C Technologie genauer bestimmt werden können.

% , nutzt unsere
% Studie stark umstrukturierte Chromosomen in \textit{Drosophila melanogaster}
% ohne pathologischen Phänotyp um die Allgemeingültigkeit dieses Mechanismus zu
% untersuchen. Zur Analyse der dreidimensionale Chromatinstruktur wurde \hic
% angewendet. Neben einer Zusammenfassung der Ergebnisse dieser Studie wird
% insbesondere die Fähigkeit der \hic Technologie, große SV sichtbar machen zu
% können, hervorgehoben.

Drittens wird der aktuelle Stand eines gemeinsamen Unterfanges vorgestellt, das
zum Ziel hat die Detektion von SV in einzelnen Zellen zu ermöglichen.
Studien zu somatischem Mosaizismus, d.h. zur genetischen Heterogenität zwischen
Zellen, konnten aufgrund technologischer Limitierungen bisher nur eingeschränkt
SV analysieren, vor allem nicht SV mit neutraler Kopienzahl oder komplexe Veränderungen.
Wir konzipierten deshalb eine neuartige Methode die, zum ersten mal überhaupt,
mindestens sieben verschiedene SV-Klassen in einzlenen Zellen ermitteln kann.
Diese Methode basiert auf drei unabhänigen Signalen die in der
Ein\-zel\-strang-Se\-quen\-zier\-me\-tho\-de Strand-seq identifiziert werden können.
Hier stelle ich eine Software mit dem Namen \mc vor, die dieses Konzept umsetzt,
und belege anhand von Beispielen deren Machbarkeit.
Um die Grenzen der Methode bestimmen zu können entwickelte ich ein vielseitig
einsetzbares Simulationsprogramm für Strand-seq-Daten und untersuchte damit die
Leistungsfähigkeit von einem der zentralen Bausteine von \mc.
Nach Fertigstellung könnte diese Arbeit Studien zu somatischem
Mosaizismus, zum Beispiel im Kontext des Alterns oder von Krbeserkrankungen,
vorantreiben.

In dieser Dissertation habe ich aufkommende Technologien benutzt um SV, vor allem
komplexe oder jene mit neutraler Kopienzahl, zu bestimmen, die mit\-hil\-fe bisheriger
Technologien nicht auffindbar waren. Dies führte zu neuen Erkenntnissen zur
Komplexität und zur funktionalen Rolle dieser SV. Darüberhinaus habe ich
Computermethoden entwichelt, die unsere Fähigkeiten zur Bestimmung von SV erweitern
und die zukünftige Studien zur Rolle dieser SV in Hinsicht auf Krankheiten
ermöglichen.

%lässt sich sagen, dass aufkommende Sequenzierungsmethoden für
%lange DNA-Ab\-schnit\-te (\acl{pacbio} und \acl{ont}) sowie neue Protokolle wie
%\hic oder Strand-seq einen enormen Fortschritt für die Detektion von SV bedeuten
%und damit den Grundstein legen für weiter Untersuchungen zur Rolle genetischer
%Variation im gesundem Menschen oder in Hinsicht auf Krankheiten.  Diese Methoden könnten
%zukünftig bisher ungeahnte Muster von SV im menschlichen Genom und in Spezies
%mit noch komplexeren Genomen enthüllen.
