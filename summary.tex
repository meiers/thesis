\cleardoublepage
\phantomsection
\chapter*{Summary}
\addcontentsline{toc}{chapter}{Summary (English)}

\Acf{sv} alter the structure of chromosomes by deleting, duplicating or
otherwise rearranging pieces of DNA. They contribute the majority of nucleotide
differences between humans and are known to play causal roles in many diseases.
Since the advance of \acf{mps} technologies, \acp{sv} have been
studied more comprehensively than ever before. However, in contrast to smaller
types of genetic variation, \sv detection is still fundamentally hampered by
the limitations of short-read sequencing that cannot sufficiently cope with the
complexity of large genomes. Emerging DNA sequencing technologies and protocols
hold the potential to overcome some of these limitations. This dissertation
presents three distinct studies each utilizing such emerging techniques to
detect, to validate and/or to characterize \acp{sv}. This work demonstrates
how \acp{sv} can be characterized that had previously been
challening, or even impossible, to assess and describes the biological insights
resulting from specific applications of the approaches.

First, inversions---a class of \sv that is notoriously difficult to
ascertain---were studied in the context of the 1000 Genomes Project. By utilizing
modern long-read sequencing technologies from both, \acl{pacbio} and \acl{ont},
hundreds of inversion loci could be validated and characterized that had remained
inconclusive in classical \acs{pcr} validation experiments. The majority of presumed
inversion loci was found to harbor additional complexity indiscernible to
low-coverage short-read sequencing. This analysis revealed an unprecedented amount of
complex variation in the human population and demonstrates the capability of
long-read sequencing to characterize complex \acp{sv}.

Second, the functional impact of large \acp{sv}, including
pericentric inversions, on gene expression was explored. Previously, a series of
studies have shown drastic effects of \acp{sv} on specific genes via a mechanism that alters
the three-dimensional conformation of DNA. While these studies concentrated on
single, pathological cases, our research project utilizes highly rearranged
chromosomes of \textit{Drosophila melanogaster} showing no pathological
phenotypes to assess the generality of such a mechanism. In order to study
chromatin structure in 3D, \acf{hic} was applied. Besides our findings on
chromatin structure, we were able highlight the capabilities of \hic to resolve
large chromosomal rearrangements and other SV classes.

Third, the current state of an ongoing project is presented with the objective
to enable-–-for the first time---the detection of a wide variety of \sv classes,
including copy-number balanced and unbalanced \acp{sv}, in single cells. Based on
single-cell stranded template sequencing (Strand-seq), at at least seven different
classes of \acp{sv} are identifiable at the single-cell level by utilizing three
independent signals present in the data. Here, the general concepts behind this
novel methodology as well as specific solutions for two out of
three steps towards this goal are presented. Furthermore, a versatile simulation
framework was established to explore the limitations of this approach.
Once completed, our method will eventually facilitate studies of \sv
heterogeneity and mosaicism, for example in the context of cancer and ageing.

In conclusion, utilizing emerging sequencing technologies such as long-read
sequencing, \hic, or Strand-seq significantly improves our abilities to assess
structural variation, which is an essential step to studying the impact of
genetic variation on health and disease. These methods may, in
the future, reveal currently hidden patterns of structural variation in the
human genome and in other species with complex genomes.




\cleardoublepage
\phantomsection
\chapter*{Zusammenfassung}%
\addcontentsline{toc}{chapter}{Zusammenfassung (Deutsch)}%

Strukturvariationen (SV) verändern die Chromosomenstruktur, in dem sie
Teile der DNA deletieren, duplizieren oder anderweitig neu anordnen. Sie sind für
den Großteil der Unterschiede in der Nukleotidsequenz zwischen Menschen verantwortlich
und sind kausal für verschiedene Er\-kran\-kun\-gen. Seit dem Vormarsch
von hochparallelen Sequenziermethoden (\textit{massively parallel sequencing},
MPS) können SV umfassender denn je untersucht werden. Dennoch leidet die Detektion von
SV, im Gegensatz zu jener von kleineren Formen genetischer Variation, unter den
Limitierungen der Sequenzierung kurzer DNA Abschnitte, welche die Komplexität
großer Genome nicht ausreichend abbilden kann. Neu aufkommende
DNA-Se\-quen\-zier\-me\-tho\-den und –protokolle bergen das Potenzial diese Limitierungen
zu überwinden. In dieser Dissertation werden drei separate Studien präsentiert,
die aufkommende Technologien zur Detektion, Validierung und/oder Charakterisierung
von SV nutzen. Diese Dissertation zeigt auf wie solche Technologien genutzt
werden können um SV zu untersuchen, die zuvor schwierig, wenn nicht gar
unmöglich, zu ermitteln waren, und erläutert die biologischen Erkenntnisse, die
unter Anwendung dieser Prinzipien gewonnen wurden.

Zunächst werden Inversionen---eine bekanntermaßen schwierig zu ermittelnde Form
von SV---im Rahmen des ``1000 Genomes Project'' untersucht. Mithilfe moderner
Sequenziertechnologie für besonders lange DNA\--Ab\-schnit\-te von \acl{pacbio}
und \acl{ont} konnten hunderte von vor\-her\-ge\-sag\-ten Inversionen validiert und
genauer cha\-rak\-te\-ri\-siert werden, welche zuvor in klassischen \acs{pcr}-basierten
Validierungsexperimenten ergebnislos blieben. Es stellte sich heraus, dass die
meis\-ten Loci mit vermuteten Inversionen tatsächlich komplexe Formen von SV
enthalten, welche mit \mps Methoden und nur niedriger Abdeckung nicht zu
erkennen sind. Diese Studie offenbarte eine ungeahnte Häufigkeit von komplexen
Inversionen in der menschlichen Population und demonstriert die Nutzbarkeit
jener Sequenziertechnologien zur Erkennung komplexer Formen von SV.

Zweitens wurde der funktionelle Einfluss von großen SV,
inklusive perizentrischer Inversionen, auf die Genexpression untersucht.
Zuvor hatten eine Reihe von Studien drastische Effekte von SV auf die
Genexpression gezeigt, welche über einen Mechanismus funktionieren, der die
dreidimensionale Struktur des Chromatins beeinflusst. Während sich diese Studien
auf einzelne Loci mit pathologischen Konsequenzen fokussieren, nutzt unsere
Studie stark umstrukturierte Chromosomen in \textit{Drosophila melanogaster}
ohne pathologischen Phänotyp um die Allgemeingültigkeit dieses Mechanismus zu
untersuchen. Zur Analyse der dreidimensionale Chromatinstruktur wurde \hic
angewendet. Neben einer Zusammenfassung der Ergebnisse dieser Studie wird
insbesondere die Fähigkeit der \hic Technologie, große SV sichtbar machen zu
können, hervorgehoben.

Drittens wird der aktuelle Stand einer laufenden Studie vorgestellt mit dem
Ziel---zum ersten Mal---ein breites Spektrum von SV-Klassen, sowohl mit unbalanzierter
als auch mit balanzierter Kopienzahl, in einzelnen Zellen zu detektieren.
Basierend auf der Ein\-zel\-strang-Se\-quen\-zier\-me\-tho\-de Strand-seq können
mindestens sieben verschiedene Klas\-sen von SV anhand drei verschiedener Signale
in Zellen bestimmt werden. Hier werden das generelle Konzept sowie
konkrete Lösungen zu zweien von drei relevanten Schritten hin zum Gesamtziel
dargelegt. Nach Fertigstellung könnte diese Arbeit Studien zu somatischem
Mosaizismus, zum Beispiel im Kontext des Alterns oder von Krbeserkrankungen,
vorantreiben.

Zusammenfassend lässt sich sagen, dass aufkommende Sequenzierungsmethoden für
lange DNA-Ab\-schnit\-te (\acl{pacbio} und \acl{ont}) sowie neue Protokolle wie
\hic oder Strand-seq einen enormen Fortschritt für die Detektion von SV bedeuten
und damit den Grundstein legen für weiter Untersuchungen zur Rolle genetischer
Variation im gesundem Menschen oder in Hinsicht auf Krankheiten.  Diese Methoden könnten
zukünftig bisher ungeahnte Muster von SV im menschlichen Genom und in Spezies
mit noch komplexeren Genomen enthüllen.
