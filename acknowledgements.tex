\cleardoublepage
\phantomsection
\chapter*{Acknowledgements}
\addcontentsline{toc}{chapter}{Acknowledgements}


First and foremost, thanks are due to my supervisor \jan, who made all of this
possible. His excitement for research and his encouraging and constructive
feedback have always been a great motivation to pursue my work. Thanks go also
to my thesis advisors Judith Zaugg, Wolfgang Huber and Benedikt Brors, who often
shared their ideas and suggestions during and outside of our regular meetings.

One of the most exciting aspects of my work was to closely collaborate with a
number of amazing people. To proceed chronologically, I would like to thank
\tobias, \markus, \andreas and \adrian for our awesome team effort in the
1000 Genomes Project. Especially Adrian, the most critical and thorough
old-school wet lab biologist I know, for looking through the 30 thousand plots
that I created. Next, I am very happy that I had the opportunity to cooperate with \yad and \alek
on our balancer project. For me, this study is the prime example of synergy
resulting from our different backgrounds, and I learned a lot about research
(and about flies) from both of them. I am also grateful to \eileen, who
initiated, supervised, and funded this project. At last, I am quite excited
about being part of the ``Strand-seq nation''. We have been pursuing exciting
research together and generally have a lot of fun during our phone calls and
hackathons! Probs to \ashley, \marschall, \david and \maryam, but also the
other ``citizens'' including \karen, \hyobin, and \venla, my Master student.

Then, there are other people who influenced my professional development more
than they might be aware. First of all, Tobias and Markus must be nominated in
this category, with whom I spent my first months at EMBL and who I have been
looking up to for their skill and their modesty to not make a big deal about it.
I don't know how often I went downstairs with a question, but I always came back
a little bit wiser and a little bit more relaxed. Another great source of
inspiration was Sebastian Waszak, with whom I discussed many scientific and
non-scientific topics, be it at EMBL or at P11. I would also like to thank
\garfield, who I often asked for counsel during the fly project and who always
gave me plenty of food for thought.

Next, I would like to acknowledge the efforts of EMBL's IT department, including
GBCS, to providing and constantly improving the infrastructure for my research.
Moreover, I thank those people who commented on this thesis, notably \jan,
\ashley, Nina Habermann, \alek and Jonas Ibn-Salem.

My time at EMBL was a thrilling, stimulating and fun experience and would not
have been the same without all the friends around me. I would hence like to say
thank you to all the members of the Korbel lab, including past members (Alexandros,
Chris, Balca and many more), for the good times we had in the lab, downtown and
on our retreats. A special thanks to Nina for lifting all the organizational
burdens from our shoulders and always having an open door (figuratively, as we
sit in the same room). I can also count myself lucky for having a nice batch of
fellow PhD students around me, including Jessi, Jørgen, Lukas, Mariana, Marvin,
and Sourabh. I am much looking forward to our next joint holiday trip!

Der größte Dank gilt jedoch meiner Familie. Auch wenn
meine Eltern bis heute nicht ganz verstehen, was ich hier genau mache, haben
sie immer an mich ge\-glaubt---danke dafür! Ebenso bin ich meinen Schwiegereltern
dankbar, die mich stets unterstützt haben. Die Unternehmungen mit der ganzen
Familie waren nicht immer erholsam, aber ein schöner und jederzeit willkommener
Ausgleich.

Zuletzt und gleichzeitig allen voran danke ich Lena und Merle. Für die
tolle Zeit die war, und die, die ganz sicher kommen wird.
