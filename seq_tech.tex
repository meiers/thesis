

%\dictum[Frederick Sanger\footnotemark]{(...) [The] knowledge of sequences could
%contribute much to our understanding of living matter.}
%\footnotetext{From his biography for the Nobel Prize\cite{Sanger1980}}

\subsection{Massively parallel sequencing}
\label{sec:mps}



\figurepagewidth{reads.pdf}{reads}{MPS sequencing reads}{Sequencing reads from
\mps are typically short (up to 150~bp), are sequenced from their $5^\prime$
ends and can be single-ended or paired-end. In paired-end or mate pair
libraries, the lengths of the original DNA fragment (which can be estimated
after read mapping) is called insert size and usually much larger in mate pair
experiments than paired-end experiments.}



\subsection{Long read sequencing}
\label{sec:long_read_seq}


\figuretextwidth{third_gen_seq_Heather2016.png}{third_gen}{Concepts of
    \texorpdfstring{\acs{pacbio}}{PacBio} and ONT sequencing}{\textit{a)} \pacbio
    sequencing utilizes the concept of sequencing by synthesis similarly to
    the Illumina technology.
    Figure taken from \citetitle{Heather2016} \citep{Heather2016} licensed under
    \acl{ccby4}.}

Pacific BioSciences\texorpdfstring{\textsuperscript{\textregistered}}{(R)} long read sequencing
Single Molecule, Real-Time Sequencing (SMRT\textsuperscript{\textregistered})

\todo{Circular consesus sequence - how it works}

Oxford Nanopore Technologies\texorpdfstring{\textsuperscript{\textregistered}}{{R}}

MinION\textregistered


\subsection{Chromatin conformation capture sequencing}
\label{sec:ccc}

\figuretextwidth{chrom_conf_cap_Li2014_CC.jpg}{ccc}{Chromatin conformation
    technologies}{Figure taken from \citetitle{Li2014} \citep{Li2014}
    licensed under \acl{ccby4}.}

\subsection{Strand-seq}
\label{sec:strandseq}

\figuretextwidth{strand_seq.pdf}{strand_seq}{Strand-seq principle}{Figure
    modified from \citetitle{Sanders2016} \citep{Sanders2016} licensed under
    \acl{ccby4}.}
