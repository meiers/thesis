\chapter{Introduction}

\section{Genetic variation}
\label{sec:variation}

\section{DNA-sequencing methodology}

\dictum[Frederick Sanger\footnotemark]{(...) [The] knowledge of sequences could
contribute much to our understanding of living matter.}
\footnotetext{\cite{Sanger1980}}

\subsection{High-throughput sequencing}
\label{sec:hts}

The foundation of DNA-sequencing as we know it today was laid in 1977 by
Sanger and his ``chain-terminaiton'' technique \citep{Sanger1977}.

\todo{cite the review ``History of sequencing DNA'' \citep{Heather2016}}

\subsection{Pacific BioSciences\textsuperscript{\textregistered} long read sequencing}
\label{sec:pacbio}
Single Molecule, Real-Time Sequencing (SMRT\textsuperscript{\textregistered})

\subsection{Oxford Nanopore Technologies\textsuperscript{\textregistered}}
\label{sec:ont}
MinION\textregistered

\subsection{Chromatin conformation capture sequencing}
\label{sec:ccc}

Hi-C \citep{Lieberman-Aiden2009} and Capture Hi-C \citep{Dryden2014,Schoenfelder2015,Jager2015,Mifsud2015}.

\subsection{Strand-seq}
\label{sec:strandseq}

\section{Structural Variation}
\label{sec:sv}
