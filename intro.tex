\chapter{Introduction}
\label{sec:intro}


\section{This thesis in a nutshell}
This thesis is about \emph{exploiting emerging DNA sequencing technologies to
study genomic rearrangements}. I summarize my different efforts at
characterizing this specific type of genetic variation and at developing the
right computational tools do to so. A common theme throughout this work is that
scientific research thrives through modern technology (DNA sequencing technology
in this case) going hand in hand with the right computational methods to analyse
and visualize data. In fact, large parts of my work describe the development of
such methods. I worked on multiple different research projects during my nearly
four years at the European Molecular Biology Laboratory, which are all connected
by this aspect of method development and the fact that they aimed at
characterizing structural variation.

As you might have realized by now, this
thesis was not written for readers who are fully ignorant of bioinformatics. I
expect a minimum knowledge of, for example, what DNA is, but I try to introduce
the more specific parts subsequently or at the beginning of each chapter.
\Cref{sec:intro} starts off by introducing genetic variation in general,
structural variation in more detail and multiple modern DNA sequencing
technologies/protocols that are essential for later chapters. At last, when I
use less well-known terms but they occur only locally or are less relevant for
the understanding of the topic I sometimes use the margin of the page to say a
few words about them.

In \cref{sec:complex_invs} I present my research on \emph{complex inversions in
the human genome}. During the final phase of the 1000 Genomes Project, in which
I participated, we discovered that most of the inversions (explained in
\cref{sec:variation}) that had been predicted are in fact not simple inversions.
Instead they are complex variants that tricked structural variation prediction
methods into calling inversions. I unravelled the true genomic structure of some
of these complex loci using long-read sequencing (introduced in
\cref{sec:pacbio,sec:ont}) and tried to gain insights on the molecular
mechanisms that might have created these classes of variation.

Chapter XXX \todo{refer to chapter} talks about Strand-seq (introduced in \cref{sec:strandseq})
and a computational approach to detect structural variation from single-cell
sequencing data. This technology nicely complements the standard DNA sequencing
as used for \cref{sec:complex_invs} as it can unravel inversions based on a
completely independent signal. I developed (or am developing) a computational
method to predict various classes of structural variation in addition to
inversions from the same Strand-seq data. A specific focus is on finding
mosaic events and on phasing those (see \cref{sec:variation}\todo{explain mosaicism in section \cref{sec:variation}}).

The functional aspect of structural variation is finally addressd in chapter
\cref{sec:balancer}. In this project I collaborated with a fly geneticist and an
expert in chromatin conformation capture analysis (explained in \cref{sec:ccc})
to study the consequences of chromosomal rearrangements on the expression of
genes and on the three-dimensional organization of chromosomes. In contrast to
a number of popular recent studies we use phenotypically healthy organisms and
rearrangements which are not known to be pathological. And in fact we
observe that gene expression appears not to be affected in the way these other
studies suggest. In this chapter, too, the development of computational methods
and visualization was instrumental to obtain results.

At the end (chapter XXX\todo{refer to last chapter}) I again summarize my
findings and draw conclusions across the multiple different projects. I again
point out how and why visualization was key in obtaining these findings and
discuss where I see the need for future improvements. The appendix
(\cref{sec:appendix}) contains supplementary information such as additional
figures and tables.



\section{Genetic variation}
\label{sec:variation}

\begin{itemize}
\item types: SNVs, SVs, aneuploidy
\item haplotypes, phasing
\item germline vs. somatic + mosaicism
\item
\end{itemize}



\section{DNA-sequencing methodology}

\dictum[Frederick Sanger\footnotemark]{(...) [The] knowledge of sequences could
contribute much to our understanding of living matter.}
\footnotetext{\cite{Sanger1980}}

\subsection{High-throughput sequencing}
\label{sec:hts}

The foundation of DNA-sequencing as we know it today was laid in 1977 by
Sanger and his ``chain-terminaiton'' technique \citep{Sanger1977}.

\todo{cite the review ``History of sequencing DNA'' \citep{Heather2016}}

\subsection{Pacific BioSciences\texorpdfstring{\textsuperscript{\textregistered}}{(R)} long read sequencing}
\label{sec:pacbio}
Single Molecule, Real-Time Sequencing (SMRT\textsuperscript{\textregistered})

\todo{Circular consesus sequence - how it works}

\subsection{Oxford Nanopore Technologies\texorpdfstring{\textsuperscript{\textregistered}}{{R}}}
\label{sec:ont}
MinION\textregistered

\todo{2d reads - how it works}

\subsection{Chromatin conformation capture sequencing}
\label{sec:ccc}

Hi-C \citep{Lieberman-Aiden2009} and Capture Hi-C \citep{Dryden2014,Schoenfelder2015,Jager2015,Mifsud2015}.

\subsection{Strand-seq}
\label{sec:strandseq}

\section{Structural Variation}
\label{sec:sv}
