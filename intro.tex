\chapter{General Introduction}
\label{sec:intro}
\todo{write general introduction}


\section{Overview of the dissertation}
\label{sec:overview}
\todo{write overview}


\section{Terminology around genetic variation}
\label{sec:variation}

Alteration in the DNA of an organism, or of a cell, can arise spontaneously via
chemical or biological processes. If not repaired faithfully, they leave
\emph{variants} (or \emph{mutations}) of the nucleotide sequence. Some variants
only affect a single nucleotide in the double-stranded DNA---these variants are
referred to as \emph{\acfp{snv}}. Others, termed \emph{\acf{sv}}
(introduced in \cref{sec:sv}), affect larger portions of DNA and are the
central topic of the research presented here.

Most metazoan organisms are diploid, which means they contain two non-identical
copies of each chromosome: one of maternal and one of paternal origin. Any new
variant (within a cell or organism) typically arises only on one of the two
homologous chromosomes. Such a variant is said to be \emph{heterozygous}. The
genomic locus harboring this variant exists in two different versions, which we
call \emph{alleles}. Specifically it harbors a \emph{reference allele}, which is
in agreement with the average individual encoded in the reference assembly, and
an \emph{alternative allele} that describes the non-reference variant form. A
site with exactly two alleles seen across a population is termed
\emph{bi-allelic}, but there are also sites that contain multiple different
alleles and are thus \emph{multi-allelic}. Chromosomes can contain many variants
and depending on the detection strategies, it is often unclear which variants
reside on the same homologue. If this is known, we refer to the ensemble of
variants present along a single homologue as a haploid genotype, or short
\emph{haplotype}. Alleles that are present in the germ line, i.e. the cells
carrying inheritable genetic material, can be propagated to offspring. This way,
an individual can end up carrying the same variant of a genomic locus on both
homologues, which makes it a \emph{homozygous} variant. Variants that are seen
more often in a population, specifically in at least 1\% of the homologoues,
are also called \emph{polymorphisms}.

When a variant is present in an individual, but not in their parents, we call it
a \emph{de novo} mutation. The variant could either have occurred within the
zygote during the first few cell divisions, or already beforehand in the
parental germ line. The latter can sometimes be inferred if other offspring of
those parents carries the same variant. Variants that occur not in the germ line
but in cells of the non-inheritable part of an organism are \emph{somatic}
variants. When an affected cell undergoes repeated divisions, a somatic variant
can be present in a relevant fraction of the total cells of an individual---this
is called \emph{somatic mosaicism}. Depending on how early in development the
variant occurred and on natural selection it can be present in all cells of the
same lineage \citep{Youssoufian2002}. Somatic variants are also the underlying
mechanism of cancer \citep{Campbell2007}.


\section{Structural Variation}
\label{sec:sv}
Genomic \acfp{sv} are commonly defined as variants affecting more than
50 consecutive base pairs of the DNA. The main purpose of this definition is to
distinguish \acp{sv} from smaller indel variants or multi-nucleotide substitutions
(i.e. blocks of consecutive \acp{snv}) \citep{Alkan2011}. Indels are variants
(of up to 50~bp) that insert or delete nucleotides, which is the same situation
seen from different perspectives (hence the neologism \textit{indel}).
A more appealing definition than the arbitrary 50~bp threshold is that indels
are detectable inside a contiguously mapped DNA sequencing read (introduced in
\cref{sec:mps}) whereas \acp{sv} are detectable across alignments, yet also
this definition no longer fully applies in the light of novel long-read
sequencing technologies (\cref{sec:long_read_seq}). Fortunately, a clear
distinction is not biologically relevant. \Acp{sv} come in many different
flavors of which the major ones are described subsequently.

\Acp{sv} in the human genome are of particular relevance for health and disease.
For example, they cause the majority of genetic differences within the human
population and are implicated in various Mendelian diseases and in cancer \citep{Weischenfeldt2013}.
Later, in \cref{sec:balancer_background} I specifically discuss the phenotypic
impact of \acp{sv} and present a study, in which I investigated a particular
aspect of the functional consequences of \acp{sv}.




\subsection{Different classes of structural variation}
\label{sec:sv_classes}

\emph{\Acfp{cnv}} describe the focal loss or gain of genetic material. They are
termed \emph{imbalanced}, as they do not leave the balance of the two homologues
intact. A loss of DNA is called a \emph{deletion}, and a gain either
\emph{duplication}, \emph{triplication} or simply by its \emph{copy number}.
For example, a deletion has a copy number of one instead of the expected copy
number of two in a diploid organism. A duplication that arises on one of the
homologues leads to total copy number of three, and so on.
Duplications are in \emph{tandem} when the additional copy inserted in direct
proximity to the original locus instead of somewhere else in the genome. The
latter is referred to as \emph{interspersed} duplication (\cref{fig:SV_classes}).
The introduction of new sequence is called an \emph{insertion}; however,
depending on the source of the incorporated DNA, insertions can be assigned to
one of several classes, only one of which is briefly mentioned later---they are
typically not counted as \acp{cnv} though.

The loss or gain of whole (or major parts of) chromosomes, historically visible
under a microscope, is summarized as \emph{aneuploidy}.
Aneuploidy can range from a single chromosome (or at least the majority of the
chromosome) being lost or gained, up to a complete increase or decrease of the
ploidy level (of all chromosomes). The expected ploidy in diploid organisms is
\explain{$2N$}{In humans $N$ equals 23, meaning that we carry 46 chromosomes
    in our cells. Interestingly this number was falsely believed to be 48 for
    three decades before it was corrected by \cite{Tijo1956}.},
where $N$ is the number of chromosomes and $2$ the number of homologous
copies. Ploidy can aberrantly increase to \emph{triploidy} ($3N$),
\emph{tetraploidy} ($4N$) or even higher states covered by the general term
\emph{polyploid}y. Cells can also be in a purely \emph{haploid} state ($1N$),
but they are rarely viable due to problems with chromosome segregation. Mixed
states, where only some chromosomes change their copy number or not all
chromosomes have the same copy number are referred to as hyperploidy
(\cref{fig:SV_classes}).

Other types of \acp{sv} do not change the total copy number of a locus. Notably,
\emph{inversions} reverse the orientation of a locus, but generally do not
include gains or losses (\cref{fig:SV_classes}). In fact, even an inversion can
introduce (or co-locate with) \ac{cnv} depending on its mechanism of formation,
which his is one of the major findings of \cref{sec:complex_invs}.
In cases where multiple \sv classes occur within the same allel we term them
\emph{complex}. The most prominent examples are \emph{inverted duplications},
which are duplications that insert in reverse orientation into the genome
(\cref{fig:SV_classes}). Nevertheless, non-complex, i.e. \emph{simple inversions}
are the prime example of balanced \acp{sv} as they re-structure the genome
without gaining or loosing genetic material.

\figuretextwidth{SVs.pdf}{SV_classes}{Types of structural variants}
    {Each case is depicted by the original locus on the left and the affected
    locus on the right, where dashed lines are used to highlight the orientation.
    \textit{Top:} Different types of focal \acp{sv} of a genomic locus (red)
    within double-stranded DNA (represented by grey line).  \textit{Bottom:}
    Chromosomes are depicted by double oval shapes. In the (balanced)
    translocation, chromosomes 12 and 17 are chosen exemplary to stress that
    exchange happens between non-homologous chromosomes. In the \loh event,
    though, the maternal and paternal homologue of the same chromosome are
    shown.}

Another class falling into the category of balanced \acp{sv} are
\emph{translocations}. For a translocation, genetic
material is exchanged between two non-homologous chromosomes. A \emph{reciprocal
translocation} is balanced because the total amount of genetic material does
not change, just the assignment of certain loci (potentially of whole arms)
to chromosomes.
But \emph{imbalanced translocations} can arise, too. Here, one chromosome remains
largely unchanged but a part of the homologue is duplicated and added to another
chromosome, which might itself loose genetic material at the same time. This can
involve whole chromosome arms, but also smaller loci---this is also covered by
the definition of translocation. Typically though, translocation refers to the
special case of a reciprocal translocation as shown in \cref{fig:SV_classes}.

Furthermore, when cells lack one of the two alleles of a larger genomic locus,
this is called a \emph{\acf{loh}}. \loh is an immdiate consequence of the
(partial) loss of a homologue, as for example happens in case of a deletion.
However, there are also copy-neutral \loh events in which the same haploid
genotype is present in two copies. This might for example occur when an
individual inherits two copies of a chromosome from one parent and none from the
other parent (uniparental disomy), but it can also occur via other mechanisms
(\cref{sec:mechanisms}). \Ac{loh} is often not observed directly, but indirectly
by looking at smaller variants (notably \acp{snv}) in a given genomic
region-–-the absence of heterozygous variants is an indicator of \loh.

Finally, various other forms of \acp{sv} exist that are of less relevance for
this work. One exception, which shall briefly be mentioned here, are
\emph{\acfp{mei}}. Mobile elements, notably \emph{transposons}, are DNA elements
that can ``jump'' within a host genome. The human consists to a large amount of
the remainders of such elements \citep{Haubold2006}, which are largely
prohibited from active transposition by repressive mechanisms in the host cell.
A \mei may occur in a cut-and-paste or a copy-and-paste fashion and, although
they principally resemble duplications or translocation, they are seen as
separate class due to the fundamentally different mechanisms of formation.





\subsection{Molecular mechanisms underlying the formation of SVs}
\label{sec:mechanisms}

In order to truly conceive \aclp{sv}, it is important to understand how they
originate. Because we understand certain mechanisms of formation, we know
today why \acp{sv} are not evenly distributed across the genome, for example, or
why they re-occur independently in specific locations \citep{Hastings2009}.
More and more accurate discovery of \acp{sv}, on the other hand, has led to
a better understanding of the functioning and impact of these mechanisms
\citep{Hastings2009,Abyzov2015}. Based on specific scars around the breakpoints
of \acp{sv}, the mechanism that introduced them can sometimes be unraveled in
retrospect. This is exactly the idea I apply in \cref{sec:complex_invs} to find
out how the complex \acp{sv} we find were formed. Here, the major molecular
mechanisms involved in formation of aneuploidy, focal copy number changes and
inversions shall hence be introduced. They have previously been described in
great detail by James Lupski and colleagues \citep{Hastings2009,Carvalho2016}.
%I explicitly exclude the pathways behind MEIs, for which I recommend \cite{Levin2011}.

Aneuploidy occurs through missegregation of single chromosomes during cell
division. During meiosis, either in oocytes or spermatozoa, this can lead to
inheritable aneuploidy, which was estimated to occur in 5\% of human pregnancies
\citep{Templado2013}. Missegregation occurs via nondisjunction of chromosomes in
meiosis I when homologues fail to separate, or in meiosis II and mitosis when
sister chromatids are not separated properly. Alternatively, it occurs as a
consequence of anaphase lag, in which a chromosome is lost in both daughter
cells due to a delayed movement of chromosomes in anaphase \citep{webAneuploidy}.
Polyploidy arises differently, for example when an egg is fertilized by two
sperm cells simultaneously or a fertilized ovum fuses with a sperm cell
\citep{webAneuploidy}. Missegregation of chromosomes occuring during mitosis may
lead to somatic aneuploidy, which is observed in many cancer types
\citep{Gordon2012}. Also, full polyploidy can occur somatically, for example via
repeated rounds of DNA replication without subsequent mitosis or with partial
mitosis without subsequent cytokinesis. This occurs naturally, as for example
in the polytene chromosomes in insect salvary glands or in hepatocytes of the
human liver, but also spontaneously as frequently seen in different types of
cancer \citep{Davoli2011}.

Focal \acp{sv} arise either during replication or after a break of the
double-stranded DNA backbone \citep{Hastings2009}. Double strand breaks and
replication errors occur stochastically or result from cellular stress, but the
cell actively counters such errors through its powerful repair mechanisms. DNA
repair is not always faithful, though, sometimes leading to the formation of
\acp{sv}. A major mechanism of \sv formation employs the homologous
recombination machinery, which uses homologous sequence (from the sister
chromatid) as a template to repair a break. However, given the repetitive
nature of the human genome, homology might not only present at the
respective locus but also in other, non-allelic loci. This \emph{\acf{nahr}}
can create various types of \acp{sv}, including deletions, duplications,
inversions and even translocation, depending on position and orientation of the
ectopic homologous sequence \citep{Carvalho2016}. The presence of homology,
notably of large segmental duplications of more than 90\% sequence identity and
several kilobases in size, predisposes the human genome to the formation of
recurrent \acp{sv} via \nahr \citep{Carvalho2016}. The other way around, when
near identical sequence is detected flanking \sv breakpoints on both ends
(20~bp are a usual lower threshold), such an \sv is believed to be formed via
\nahr \citep{Onishi-Seebacher2011}.

Other repair mechanisms do not require homology. \emph{\Acf{nhej}} is the
dominant pathway during G0/G1-phase to efficiently re-ligate the ends of DNA
double strands, usually leaving traces of not more than a few deleted or
inserted base pairs at the junction \citep{Lieber2008}. Importantly, \nhej is
available to the cell before sister chromatids are present and is a fast way to
react to double strand breaks. When multiple double strand breaks arise
simultaneously, \nhej can falsely ligate genomic loci in the wrong order and
thus introduce \acp{sv}. A related mechanism uses sequence identity of few
(as little as 1-4) base pairs, also known as \explain{micro-homology}{It is
debatable whether the term \emph{homology} is correct here, i.e. whether the
short stretches of identical DNA on both sides of an \sv breakpoint in fact
share a common evolutionary ancestry}, to initiate re-ligation via a mechanism
called \emph{micro-homology-mediated end joining} \citep{Hastings2009}.

Replication of DNA, which happens prior to each cell division, is also
susceptible to errors. For example, through a process called \textit{replication
slippage} smaller deletions and duplications can arise between stretches of
homology within a replication fork, limited by the size of an Okazaki fragment
\citep{Hastings2009}. Furthermore, the DNA backbone can break within a
replication fork, leaving a single-ended double strand break. Such a break can
be faithfully resolved by the \emph{\acf{bir}} mechanism: a single strand of the
unfinished DNA molecule anneals to homologous sequence in the template DNA to
restart replication, which can continue up to hundreds of kilobases from there
\citep{Carvalho2016}. Again, this search for homology may fail and either anneal
to the homologous chromosome (instead of the sister chromatid), leading to
extended stretches of \loh, or ectopically, resulting in one of several possible
\sv types including \acp{cnv} and inversions.

Other replicative mechanisms of \sv formation requires no, or only short
stretches of micro-homology. Notably, a version of \bir that can operate
independent of the homologous recombination machinerey was described, which
relies only on micro-homology (4-15bp) to invade template DNA and the mechanism
of which was consequently called \emph{\acf{mmbir}} \citep{Hastings2009a}.
Moreover, homology-independent rearrangements occurring during replication can
include multiple complex rearrangements and are more prone to copy gains than
losses, in concordance with a model of \acf{fostes} \citep{Zhang2009a,Hastings2009}.



\section{DNA sequencing technologies}
\label{sec:sequencing}
DNA sequencing refers to the process of deciphering the order of the four bases
(adenine, cytosine, guanine, and thymine, abbreviated by A, C, G, and T) that
constitute a DNA molecule. In the 1990s, sequencing the human DNA was a
decade-long, multi-million dollar effort but it led to the successful production
of a reference genome of humans and many other organisms
\citep{Lander2001,Venter2001}. Today, thanks to technological improvements,
DNA sequencing has become a standard technique applied on a daily basis in
genomics research. The advance of sequencing technologies has truly
revolutionized genetic research and brought unforeseen capabilities also to
studies of structural variation. In fact, these capabilities are not yet completely satisfactory
(as described in \cref{sec:sv_detection}) and ongoing
development of new technologies and protocols is continuously pushing the
boundaries of what is possible. Since DNA sequencing takes such a prominent
position in my research---virtually every experiment in this thesis includes
sequencing---the major techniques shall be introduced here.




\subsection{Massively parallel sequencing}
\label{sec:mps}

The foundation of modern DNA sequencing technologies was laid in 1977 by Frederick Sanger
and his ``chain termination'' technique \citep{Sanger1977}. Despite not being
the first DNA sequencing method, the chain termination method brought
unprecedented ease of use and accuracy \citep{Heather2016}. It is based on DNA
polymer extension via DNA-dependent DNA polymerase (i.e. replication) and the
incorporation of dideoxynucleotides, which stop polymerization. With subsequent
electrophoresis, partially replicated DNA fragments can be ordered by length and
nucleotides identified based on radioactive or fluorescent labels.
Sanger sequencing was instrumental in the Human Genome Project and, owing to the
accuracy and length (around 1~kb) of sequenced fragments, it is still used
today for validation purposes. It sequences single DNA fragments, though, and
is thus laborious to apply in a larger scale.

The throughput could be dramatically increased with the advance of \acf{mps},
which is also referred to as short-read sequencing, next-generation sequencing,
2\textsuperscript{nd}-generation sequencing, or high-throughput sequencing.
Different commercial techniques were brought forward in the first decade of this
millennium, including Pyrosequencing by 454 Life Sciences, Sequencing by
Oligonucleotide Ligation and Detection (SOLiD) by Applied Biosystems,
Nanoball Sequencing by Complete Genomics, Helicos Single Molecule Fluorescent
Sequencing by Helicos BioSciences, and the Reversible Terminator Chemistry by
Solexa \citep{wikiDNAseq2018}. Today, the market for \mps technologies is vastly
dominated by Illumina, who acquired Solexa and their technology in 2007. Here,
the core principles of Illumina DNA sequencing shall be described representative
for \mps in general.

\paragraph{Key principles of parallel DNA sequencing}
Like the Sanger technique, Solexa/Illumina’s approach relies on the concept of
sequencing by synthesis, i.e. by replication through a DNA polymerase. It in
fact also utilizes the incorporation of fluorescently labeled dideoxynucleotides,
which initially terminate the polymerization. A major novelty, though, is that
the fluorescent label can be removed and the $3^\prime$ hydroxyl group of the
dideoxynucleotide chemically restored. This technique is widely known as
reversible terminator chemistry \citep{Turcatti2008}. DNA is then replicated
step by step, in each of which the incorporated nucleotides are detected using
fluorescent imaging. This concept of cyclic DNA synthesis followed by
fluorescent detection is shared by multiple of the aforementioned techniques,
which use slighly different molecular mechanisms \citep{Shendure2008}.

Another key concept of \mps technologies is a step of clonal amplification of
DNA fragments in order to enhance the fluorescent signal detection. DNA
fragments are initially ligated to adapter sequences and then, in case of
Solexa/Illumina, immobilized on the flow cell and amplified via
\explain{\acf{pcr}}{A method for amplification of DNA fragments. Did you know
    that \acs{pcr} was invented only after Sanger
    sequencing, in 1983? See \citet{Mullis1990} for a brief history}
    \citep{Mullis1990}, which they call bridge amplification.
The aspect of parallelism comes into play when many (up to millions of) local
clusters, each with a clonally amplified DNA fragment, are observed
simultaneously during nucleotide incorporation. This again was driven by
technological advances in high-resolution cameras, notably based on
charge-coupled devices \citep{Barbe1975,Shendure2008}. Due to clonal
amplificaiton and high-resolution imaging, Illumina machines can sequence DNA
with extremely high accuracy, with a per-base error rate is in the order
of 0.1\% \citep{Fox2014}.

\paragraph{Applications}

\figurepagewidth[t]{reads.pdf}{reads}{MPS sequencing reads}{Sequencing reads
    from \mps are typically short (for example 100~bp), are sequenced from
    their $5^\prime$ ends and can be single-ended or paired-end. In
    paired-end or mate pair libraries, the lengths of the original DNA fragment
    (which can be estimated after read mapping) is called insert size and
    usually much larger in mate pair experiments than in paired-end experiments.}

In contrast to Sanger sequencing, which targets a single locus, \mps can be
applied to perform \emph{\acf{wgs}}. During \wgs, DNA is highly fragmented prior
to the construction of a sequencing \emph{library}, which is then sequenced via \mps to
yield a large set of \emph{sequencing reads} from
all over the genome. Due to the random fragmentation this approach is also
commonly known as shotgun-sequencing \citep{Weber1997}. In a typical
\emph{re-sequencing} experiment, where a species with available reference genome
is sequenced, these reads are then mapped to the reference for further analysis
such as, for example, variant detection.

Apart from \wgs, a large number of
other sequencing protocols exist that utilize \mps to study different molecular
characteristics\footnote{see \citet{Pachter2018} for a list of such protocols}.
A prominent example is RNA-seq \citep{Morin2008,Wang2009},
which makes the mRNA present in cells available to sequencing by
reverse-transcription into cDNA. We used this technique in \cref{sec:balancer}.
\Cref{sec:ccc,sec:strandseq} cover two other protocols based on \mps that are
of particular interest in this work.

\paragraph{Paired-end sequencing}
Modern sequencing machines offer the possibility to sequence a DNA fragment
from both ends. In case of Illumina/Solexa, this is achieved by a special step
of bridge amplification that anneals the free end of all fragments in a clonal
cluster to the surface and then frees the initially attached ends. Afterwards,
sequencing continues in opposite direction to capture the other end of the
DNA fragments. This approach is called \emph{paired-end} sequencing or
paired-end tag sequencing and was used early on to study structural variants
\citep{Campbell2008} (see also \cref{sec:sv_detection_ngs}). Using pairs, more
bases can be sequenced at high quality than could be with a single read.
Typically, the DNA fragments subject ot paired-end sequencing have a length of
up to 500~bp. Larger fragments (typically around 3~kb, but up to 10~kb) can be
achieved by creating \emph{mate pair} libraries (a.k.a jumping libraries). In the
approach of mate pair sequencing, a longer DNA fragment is first circularized before
the connection of both ends is sequenced either single-ended or in paired-end mode
\citep{Korbel2007}. The size of the underlying DNA fragment is called
\emph{insert size}, and the number of sequenced bases \emph{read length}
(\cref{fig:reads}). Paired-end and mate pair sequencing have played a pivotal
role in \sv detection, which is elaborated later in this introduction.

\paragraph{Sequence analysis}
After DNA sequencing, the computational analysis of the obtained sequencing
reads begins. Naturally, this analysis may be very different depending on which
protocol was used. For a \wgs resequencing experiment, a very common first step
is to assign the short reads to their most likely origin and
\explain{read orientation}{Only one DNA strand (e.g. the $5^\prime$ strand) is
    encoded in a reference genome, but fragments from both strands are sequenced.
    Thus, also the reverse complement sequence of each read must be mapped---we
    say they are mapped to the \emph{minus} strand (\cref{fig:reads})} within the reference
genome. This process is called \emph{read mapping} or \emph{read alignment}, and
software tools to perform this task are abundant
\citep{Li2009,Weese2009,Langmead2009,Alkan2009,Li2013}. The intricacies that
hamper read mapping are sequencing errors and the repetitive nature of large
genomes, which do not allow unique placement of reads in many regions. These
regions are said to have low \emph{mappability} and are difficult to deal with---often they are simply neglected.
Subsequent to read mapping, downstream analyses can be carried out such as \sv
detection, which I describe in \cref{sec:sv_detection}.

The very popular paired-end sequencing allows the mapping of a read to be
informed by the placement of the second read. For example, when one read maps
ambigously, it can somtimes be ``anchored'' by the second, uniquely mapping read.
Paired-end reads, which are sequenced from their $5^\prime$ ends towards one
another, align in convergent orientation to the reference genome, whereas mate
pairs align in divergent orientation (\cref{fig:reads}).

An alternative to read mapping is \emph{de novo} assembly, which exploits the
relationship of sequencing reads to one another (i.e. a common subsequence) to
restore the sequenced genome independent of a reference. These methods, however,
typically fail to produce long consecutive sequences \citep{Alkan2011_assembly}.










\subsection{Emerging long read sequencing technologies}
\label{sec:long_read_seq}

\figuretextwidth[t]{third_gen_seq_Heather2016.png}{third_gen}{Concepts of
    \texorpdfstring{\acs{pacbio}}{PacBio} and \texorpdfstring{\acs{ont}}{ONT}
    sequencing}{
    \subpanel{A} DNA polymerase inside a zero-mode waveguide processes a DNA
    molecule. The fluorescently labeled nucleotides are preferably illuminated
    at the bottom of the well, distinguishing their signal from the pool of
    nucleotides in the solution.
    \subpanel{B} Nanopore sequencing by passaging of a single-stranded DNA
    polymer through a pore in a non-conductive layer. Movement of the DNA
    molecule is facilitated by a voltage across the layer and decelerated by a
    processive enzyme (upper ellipse), e.g. a helicase that unwinds the double
    stranded DNA.
    Figure taken from \citetitle{Heather2016} \citep{Heather2016} licensed under
    \acl{ccby4}.}

Over the last years, new sequencing technologies have been developed which are
commonly referred to as ``3rd-generation sequencing'' in the community. These
technologies are fundamentally different to \mps technologies as they avoid
clonal expansion of DNA fragments, but sequence single molecules instead.
Although single molecule sequencing had been feasible already earlier
\citep{Braslavsky2003}, later commercialized by Helicos BioSciences,
third-generation sequencing is commonly associated only with the techniques of
\acf{pacbio} and \acf{ont}. In contrast to the Helicos platform, these
techniques achieve significantly longer read lengths (of up to more than 100~kb)
at usually decreased accuracy. Unlike Sanger's technology, they still sequence
many molecules in parallel.

The technology of \acl{pacbio} relies on the concept of sequencing by synthesis using
fluorescently labeled deoxynucleotides. Unlike with Illumina, the sequencing
occurs in real time on single molecules and was hence termed
\emph{Single-Molecule Real-Time sequencing (SMRT)} \citep{Eid2009}. Fluorescent
image detection occurs not in cycles, but continuously (creating a \emph{movie}).
Besides deciphering the order of nucleotides, this principle also measures
kinetics of the polymerase, allowing the detection of modified bases such as
methylated cytosine \citep{Flusberg2010}. The major challenge in SMRT sequencing
is the detection of the fluorescence signal from a single nucleotide upon
ligation. To this end, researchers had engineered microplates with so-called
zero-mode wave guides \citep{Uemura2010}. The laser for excitation of
fluorophores only illuminates the bottom of these nanowells, in each of which a
polymerase is deposited so that nucleotides can be detected during incorporation
against a background of fluorophores outside the well \citep{Heather2016} (see
\cref{fig:third_gen}a). Single-molecule base detection is still noisier than
detection in clonally amplified sequences, leading to reported per-base-error
rates of 11-15\% \citep{Rhoads2015}. To improve accuracy, \pacbio researchers
promoted a technique called \emph{circular consensus sequence (CCS)}
\citep{Travers2010}. Here, a double-stranded DNA fragment ligates to circular
adapters on both ends to form a circular DNA backbone, which still preserving
its DNA double-stranded structure. The polymerase can then pass this ring of DNA
repeatedly, effectively sequencing the same fragment multiple times. This long
read is computationally divided into sub-reads and a consensus is formed with
reported accuracies of up to more than 99\% \citep{Rhoads2015}.

\Acl{ont} utilize a fundamentally different approach to sequence single
molecules. Driven by electrophoresis, a single-stranded DNA passes through a
tiny pore that can detect changes in the ionic current specific to the type of
nucleotide passing through. The development of this technique spanned three
decades and is based on findings from multiple labs, as \citet{Deamer2016}
nicely portrayed. One major step towards the current technology was to find an
appropriate pore that is just wide enough for a single-stranded DNA molecule to
pass through: Initially \textit{α-hemolysin} channels from
\textit{Staphylococcus aureus} had been used, which were later replaced by a
genetically modified version of \textit{MspA}, a porin from
\textit{Mycobacterium smegmatis} that allowed a much better signal-to-noise
ratio \citep{Butler2008}. A second crucial step was to decelerate the passage of
each single nucleotide through the pore in order to allow accurate measurements
of currents. While DNA polymers would pass a pore at a rate of $10^{-13}$ sec
per nucleotide even for the smallest voltages, researchers had the idea to place
processive enzymes in front of the pore that slightly pause passage with each
nucleotide \citep{Deamer2016}. \Ac{ont} offered the first commercially availabe
nanopore sequeqncing devices, named MinION, in an early access program from 2014.
They utilize an ATP-dependent helicase enzyme that unwinds double-stranded DNA
before passaging the pores, with detection happening simultaneously at up to
512 pores \citep{Jain2015}. A hairpin adapter is ligated to DNA fragments, so
that after a single strand has passed the pore the complementary strand
(linked covalently via the hairpin adapter) follows. This sequences the same
fragment twice and is utilized to improve sequencing quality. The corrected reads
are called \emph{2D reads} and were reported to have an improved error rate of
15\% \citep{Jain2015}. Through the MinION early acces program our laboratory
gained access to MinION devices, so that we could apply the technique in my
research project describe in \cref{sec:complex_invs}.






\subsection{Chromatin conformation capture sequencing}
\label{sec:ccc}

Studies of the three-dimensional structure of in vivo chromatin have for a long
time been limited to imaging-based approaches. \citet{Dekker2002}, however,
proposed a new technique named \emph{chromatin conformation capture (3C)} that
probes the three-dimensional distance of loci using genomics methods. The basic
idea is to crosslink DNA with itself in regions that are in close spatial
proximity. The results of such experiments give insight into the contact
frequency between two loci relative to other loci, which can be interpreted as
an average three-dimensional distance between the loci observed across many
nuclei. The original 3C protocol relies on targeted \pcr amplification and is
capable of testing the interaction between exactly two loci. However, multiple
protocols based on 3C have recently been shown to increase the level of
parallelism, including 4C \citep{Zhao2006,Simonis2006} and 5C \citep{Dostie2006}.
Eventually, the combination of chromatin conformation capture and \mps resulted
in \acf{hic} \citep{Lieberman-Aiden2009}. The core principle of \hic, as shown
in \cref{fig:ccc}, involves cross-linking and digestion of DNA and then
submitting cross-linked loci to paired-end \mps. After filtering of data (e.g.
removing fragments ligated to themselves) both reads of a pair represent two
loci that were in close three-dimensional proximity within the nucleus.
Principally, the unbiased contact frequency of all loci against all other loci
can be measured in this way. In practice, the resolution of these two-dimensional
contact map strongly depends on sequencing coverage. For example, to achieve
1~kb resolution, \citet{Rao2014} utilized 4.9 billion pairwise contacts in a
human cell line.

\figuretextwidth{chrom_conf_cap_Li2014_CC.jpg}{ccc}{Chromatin conformation
    capture technologies}{
    Chromatin conformation capture relies on cross-linking of DNA, enzymatic
    digestion and subsequent ligation of cross-linked loci. In the 3C-protocol,
    targeted interactions can be analyzed via \pcr. The \hic protocol enriches
    for cross-linked fragments based on biotin labels before these fragments
    are paired-end sequneced. Figure modified from \citetitle{Li2014}
    \citep{Li2014}  licensed under \acl{ccby4}.}

\hic was designed and used to study chromatin conformation and specific
three-dimensional features such as DNA loops forming between enhancers and
promoters. It further revealed previously unknown structural features of the
genome, which are discussed in more detail in \cref{sec:balancer_hic_svs}.
However, the characteristics of \hic contact maps predestinate them for at least
two additional use cases: Intra-chromosomal interactions are more frequent than
inter-chromosomal interactions, which is in concordance with the theory of
chromosomal territories \citep{Cremer2001}. Thus, \hic data can by utilized to
cluster genomic loci by chromosome, which is especially relevant for \textit{de
novo} assembly \citep{Burton2013}. Another observation is that the contact
frequency between loci typically decays quickly with increasing genomic
distance, meaning that the highest signal is detected between loci that are also
close in linear genomic proximity. Because of that, larger genomic
rearrangements become prominent in contact maps---I utilized this feature of
\hic maps to characterize genomic rearrangements in \cref{sec:balancer}.





\subsection{Strand-seq}
\label{sec:strandseq}

Strand-seq is a single-cell DNA sequencing protocol that preserves the identity
of homologues by sequencing only the template strand of each chromosome
\citep{Falconer2012,Sanders2017}. The readout are sequencing reads, obtained via
\mps (e.g. on an Illumina platform) in either paired-end or single-ended mode,
which all map in the same orientation (plus strand or minus strand) to a
reference genome if they originated from the same homologue. In cells where
homologues are inherited on opposite strands, so-called Watson and Crick
strands, the original homologue for each read (including potential variants
captured by this read) can be determined simply based
on its mapping orientation. Strand-seq can thus reliably phase (i.e. distinguish
haplotypes) chromosomes in their full length \citep{Porubsky2016}. Moreover,
given the consistent directionality of sequencing reads across a homologue,
Strand-seq allows to detect (large) inversions in respect to the reference
genome \citep{Sanders2016}.
Strand-seq was futher used to study sister chromatid exchange events
\citep{Falconer2012}. In \cref{sec:mosaicatcher}, I present a novel approach
that utilizes Strand-seq libraries across multiple single cells to detect mosaic
\acp{sv}.

\figuretextwidth{strand_seq.pdf}{strand_seq}{Strand-seq principle}{
    \subpanel{A} Diploid cells (here schematically represented only for a single pair of
    chromosomes) contain maternal (M) and paternal (P) homologous chromosomes,
    each of which is a double-stranded DNA molecule. Watson (W) and Crick (C)
    are highlighted in orange and green. After replication in the presence of
    BrdU instead of Thymidine (1), cells contain two sister chromatids of each
    homologous chromosome, each with a different strand labeled (dotted line).
    After cell division, photolytic nicking of BrdU sites and library
    preparation (2), only sequencing reads from the non-labeled strand remain.
    \subpanel{B} A daughter cell inherits the two homologues in one of
    four different constellations, i.e. both as W (top left), both as C (top
    right), or both as different strands (maternal W, bottom left, or paternal
    W, bottom right). After read mapping cells appear in WW, CC, or WC
    configuration (circles). Figure modified from \citetitle{Sanders2016}
    \citep{Sanders2016} licensed under \acl{ccby4}.}

Strand-seq requires actively replicating cells, which are grown for a single
round of replication in a $5$-Bromo-$2^\prime$-deoxyuridine (BrdU) medium.
The incorporation of this thymidine analog into the newly synthesized DNA strand
is the basis for obtaining stranded sequencing libraries, because after cell
division each daughter cell will have only one strand labeled with BrdU
(\cref{fig:strand_seq}). After cytokinesis, the DNA of a daughter cell is
enzymatically digested into fragments using a micrococcal nuclease enzyme
(MNase). These fragments are ligated to sequencing adaptors. BrdU
containing fragments are then degraded via photolytic cleavage, so that
subsequently only non-BrdU containing fragments are amplified via \pcr.
Fragments from each single cell are tagged with cell-specific barcodes and are
finally sequenced simultaneously, typically 96 cells together in one Illumina
lane \citep{Sanders2017}.





\section{Structural variant detection}
\label{sec:sv_detection}
\subsection{Traditional SV discovery}
\label{sec:sv_detection_old}

A traditional way to investigate the karyotype of cells is by arresting cells in
metaphase, staining chromosomes and observing them under a microscope
\citep{Speicher2005}. The images of each chromosome are then ordered by
chromosome to show an ideogram. By counting the number of each chromosome in the
cell, gross abnormalities such as aneuploidy can be detected, which is
schematically depicted under aneuploidy in \cref{fig:SV_classes}. Relevant
disease-linked forms of aneuploidy could be unraveled early on this way, e.g.
the trisomy of chromosome 21 \citep{Lejeune1959}. The classic technique has been
refined to allow higher specificity and resolution, mostly via improved ways of
staining, such as by quinacrine staining, Giemsa banding, or chromosome-specific
labeling based on in situ hybridization \citep{Speicher2005}. These techniques
principally allow the detection of \acp{cnv}, inversions and translocations,
yet the size range of these \acp{sv} has to be in the order of several Megabases
or larger. Fluorescence in situ hybridization \citep{Bauman1980}, which relies
on the annealing of fluorescently labeled DNA probes to their complement DNA,
is also applied in a targeted manner for validating predicted SVs of slightly
smaller size.

Other means of detecting \acp{sv} include optical mapping
\citep{Schwartz1993,Teague2010} and hybridization-based microarrays. The latter
ones, which are reviewed in \cite{Alkan2011}, used to be the dominating method
for \cnv detection before high-throughput sequencing became standard. One of the
two major techniques in this category, namely array comparative genomic
hybridization, utilizes the competition of the test sample's DNA and a reference
DNA to a hybridization probe (e.g. short oligonucleotides) to infer the relative
copy number of the tested locus \citep{Snijders2001}. Using high-density arrays,
this method can successfully detect deletions down to 500~bp in size
\citep{Conrad2010}. SNP arrays, on the other hand, utilize hybridization probes
at sites of polymorphic \acp{snv} to measure the allelic ratio within a single
sample, which is called \emph{\acf{baf}}. This way, \acp{cnv} but also \loh can
be detected.





\subsection{SV discovery in the era of massively parallel sequencing}
\label{sec:sv_detection_ngs}

Today, aforementioned techniques have largely been superseded by \sv detection
utilizing \acl{mps} data.
% although especially microarrays remain a cost-effective alternative.
\Ac{sv} detection methods based on \mps are often (also by \cite{Alkan2011})
separated into four different conceptual approaches, namely read pair analysis,
split-read analysis, read depth analysis, and sequence assembly. In practice,
\sv prediction tools do not necessarily fit into only one of these categories.
Below I will summarize the major ideas behind the different approaches as well
as representative software implementing them.

\paragraph{Paired-end analysis}
\Ac{sv} detection based on paired-end sequencing utilizes the orientation and
expected distance of two sequencing reads to another to detect rearrangements.
For instance, when their mapping distance on a reference genome is larger than
expected a deletion may have occurred in the test sample anywhere in between
those reads. Paired-end analysis can principally detect many different types of
\acp{sv}, including \acp{cnv}, inversions, translocations, insertions and
\acp{mei} (\cref{fig:SV_detection}, B). This technique had first been used on
bacterial artificial chromosome
\citep{Volik2003}, on fosmid libraries \citep{Tuzun2005} and then on mate pair
sequencing of human genomes \citep{Korbel2007}. It is nowadays one of the
dominating principles of \sv detection and is implemented in well-established
software tools such as \textsc{BreakDancer} \citep{Chen2009}, \delly,
\textsc{CLEVER} \citep{Marschall2012}, or \textsc{LUMPY} \citep{Layer2014}.
Besides the richness in detectable SV classes, an advantage of the paired-end
read signature is that it can identify the breakpoints of an \sv. The breakpoint
accuracy depends on sequencing coverage and insert size distribution, but
it is typically in the range of few to several hundred base pairs.

\paragraph{Split-read analysis}
Split-read approaches utilize the fact that sequencing reads, if long enough,
can be divided and separately assigned to different locations of the
reference assembly  (\cref{fig:SV_detection}, C). This is different from intra-read
gaps or mismatches, which are still tolerated within each alignment and
which are typically used to detect \acp{snv} and small indels.
Based on the position and orientation of the partial alignments, split-read
analysis detects \acp{sv} in the same way as paired-end analysis. The major
advantage, though, is that breakpoints can be determined much more accurately,
often down to the exact nucleotide.
Split-read approaches had been explored, for instance, early on during the
1000 Genomes Project (\cref{sec:1000G}) on 400~bp single-ended reads
\citep{Zhang2011}. Tools such as \textsc{Pindel} \citep{Ye2009} or
\textsc{BreakSeq} \citep{Lam2010} were among the first ones to specifically
implemented the split-read approach. Nowadays, strongly encouraged by increasing
read lengths form standard MPS machines (e.g. up to 2~x~300~bp on an Illumina
MiSeq platform nowadays), read mapping software has been refined towards the
ability to directly perform split-read mapping, as exemplified by tools such as
the widely used \bwamem or specialized tools like \textsc{YAHA}
\citep{Faust2012} and \textsc{SplazerS} \citep{Emde2012}.
This allowed other popular paired-end analysis detection tools to incorporate
the split-read approach, for example in \delly, \textsc{MATE-CLEVER}
\citep{Marschall2013}, and \textsc{LUMPY}.

\figuretextwidth[t]{SV_detection_methods_Tattini2015_new.png}{SV_detection}{
    Principles of SV detection in MPS data}{Schematic representation of the four
    distinct mechanisms for SV detection on three examples:  A deletion,
    an insertion (of duplicated sequence, for example), and an inversion.
    Figure modified from \citetitle{Tattini2015} \citep{Tattini2015} licensed
    under \acl{ccby4}.}

\paragraph{Read depth analysis}
A complementary method for detecting \acp{cnv} utilizes the total read depth
signal inside an \sv  (\cref{fig:SV_detection}, A). This resembles the
methodology of microarrays, yet with
improved resolution since all of the (mappable) genome can be covered instead
of selected loci. \textsc{SegSeq} \citep{Chiang2009} and \textsc{CNV-seq}
\citep{Xie2009} were among the first tools that utilized read depth in a
sample-vs.-control scenario to detect \acp{cnv}. \textsc{mrFast}
\citep{Alkan2009} and \textsc{CopySeq} \citep{Waszak2010} extended this
approach to single-sample \cnv calling, and later \textsc{CNVnator}
\citep{Abyzov2011} and \textsc{genome STRiP} \citep{Handsaker2015} gained more
popularity. Normalization of read depth is a major challenge in these approaches
owing to an uneven sequencing coverage, which is why these tools typically
perform best when an internal reference (e.g. a control sample) is available.
The currently popular population-scale \cnv caller \textsc{genome STRiP} even
suggests a minimum of 20 to 30 sample genomes in order to perform well. In
contrast to paired-end or split-read detection, the read depth method cannot
accurately predict breakpoints, which becomes a major disadvantage especially
for smaller variants. For very large variants, on the other hand, read-depth
methods can still robustly predict \acp{cnv} even when their breakpoints reside
in repetitive regions.

\paragraph{Sequence assembly} At last, sequence assembly-based methods do not rely on the
information provided by read mapping software but perform \textit{de novo}
assembly instead, as demonstrated for example by \citep{Li2011a}. A comparison,
e.g. via sequence alignment, of the sample sequence to the reference genome then
reveals the presence of \acp{sv}  (\cref{fig:SV_detection}, D).
While whole-genome assembly still remains computationally expensive
\citep{Bradnam2013}, there are methods that perform local re-assembly of reads,
for example \textsc{TIGRA} \citep{Chen2014b} or the recent \textsc{novoBreak}
\citep{Chong2017}. Especially for \sv types that are difficult to detect by
paired-end or split-read approaches, notably for insertion of novel DNA sequence,
assembly can yield a great benefit. Tools that address this are \textsc{NovelSeq}
\citep{Hajirasouliha2010}, \textsc{MindTheGap} \citep{Rizk2014}, and
\textsc{Basil/Anise} \citep{Holtgrewe2015}.

In practice, \sv detection using any of the four different approaches typically
requires an additional filtering step after initial \sv prediction. Such
filters can rely on the quality metrics provided by the prediction tool, but
optimally they involve an independent signal, such as read coverage for
paired-end predicted \acp{cnv}. One useful signal that shall be highlighted here
is \acl{baf}---this idea from microarrays is applicable to \mps data,
too. At sites of heterozygous \acp{snv} that reside within a putative \sv, the
sequencing reads supporting both alleles can be contrasted to infer the copy
number of the locus. Notably, in \cref{sec:balancer_cnv} this principle was
instrumental to validate predicted duplications.





\subsection{State of the art and limitations of SV studies}
\label{sec:limitations}

\Acl{sv} in the human genome have been studied many times, driven by the
availability of new technology. Initial population studies mapped large \acp{cnv}
in several individuals using microarrays \citep{Sebat2004,Iafrate2004,Sharp2005,Redon2006}.
With refinements of these techniques, \cnv discovery could later be expanded to
hundreds of individuals and down to a detection size of 1~kb (or even 500~bp),
which scaled up the number of detected variants tremendously
\citep{McCarroll2008,Conrad2010}. Further improvements in the discernible size
range, the breakpoint accuracy and the types of SVs detectable were reached
with the application of \mps technologies. \cite{Korbel2007} and \cite{Kidd2008}
were among the first studies to utilize paired-end-like approaches to study
\acp{sv}, including balanced ones, in few individuals. A series of studies
followed that explored all the different technical approaches mentioned in
\cref{sec:sv_detection_ngs}.

The first phase of the 1000 Genomes Project presented the then most
comprehensive \sv call set, which was based on on low-coverage sequencing data
of 179 individuals \citep{Mills2011}. However, especially inversion detection
faced major limitations within the project, as I describe in more detail in
\cref{sec:complex_invs}. Further studies continued \sv characterization in the
human population (e.g. \citep{Sudmant2015a,Hehir-Kwa2016}) and in disease,
revealing thousands of copy number variable loci that are linked to pathological
phenotypes \citep{Swaminathan2012,Forbes2011}. \Acp{sv} were also mapped extensively in
cancer genomes \citep{Weischenfeldt2016,Campbell2017} and in other organisms such
as \textit{Drosophila melanogaster} \citep{Massouras2012,Zichner2013} or
\textit{C. elegans} \citep{Maydan2010}.

\figuretextwidth[t]{SV_limitations.pdf}{sv_limitations}{Examples for limitations
    of MPS-based SV detection}{Three examples for cases in which current
    \mps-based \sv detection methods fail. The upper half in each panel shows
    sample DNA carrying an \sv; below is shown where reads from that sample map
    to the reference assembly. \textbf{A:} A duplication of a repetitive element
    (shaded boxes) occurred, which is not detected because read mapping is masked
    within the repetitive region. \textbf{B:} An inversion flanked by repetitive
    elements. Because paired-end reads cannot be mapped uniquely inside the
    repeats, this inversion remains undetected. With an read length or insert
    size larger than the size of the repeats the inversion could be revealed.
    \textbf{C:} Insertion of a small piece of DNA into another chromosome leads
    to the prediction of a reciprocal trasnlocation. The fundamental principal
    behind this limitation is that standard \mps-based calling only detects the
    breakpoints of a copy-neutral rearrangement, but cannot reason about their
    inner state. This is different in a \cnv, for which the inner state (i.e.
    its read coverage) can be utilized for calling. The issue depicted here
    also arises for other \sv classes, notably inversions.}
\todo{harmonize the way I refer to subfigures throughout the thesis (\textbf{A:}, or \textbf{a)}, or \textit{A)},...)}

Increases in sequencing depth improved breakpoint accuracy and sensitivity of
\sv detection, yet it did not overcome specific limitations owed to the
repetitive nature of the human genome. Notably, \acp{sv} attributed to \nahr are
known to be flanked by repeat sequence, in which read mapping (and consequently
paired-end \sv detection) often fails. This has been termed the ``short-read
dilemma'' \citep{Onishi-Seebacher2011}. Unfortunately, the human genome consists
to a large portion of repeats. Sequence analyses found that up to two third of
the human genome are derived from repetitive elements (mostly transposable
elements) \citep{DeKoning2011} with around 5\% of the genome containing large
(>10~kb) and highly identical segmental duplications \citep{Lander2001}.
Especially repeat-embedded inversions cannot be detected based on traditional
techniques (including \mps) ``at high throughput and high resolution''
\citep{Sanders2016}. \Cref{fig:sv_limitations} depicts three exemplatory
scenarios in which repeats confuse paired-end sequencing based \sv detection.

Challenges related to the repetitive nature of our genomes have been discussed
extensively in other areas, notably for de novo assembly \citep{Alkan2011_assembly}
and haplotype phasing \citep{Browning2011}. In both cases, blocks of information
(e.g. phasing blocks or contigs) fail to span through repetitive genomic regions.
These challenges are even greater in other species with more repetitive or
polyploidy genomes, which tend to also have of reference assemblies of lower
quality. Livestock and crop are two examples with outstanding environmental
relevance that suffer from these limitations \citep{Bickhart2014,Saxena2014}.

\todo{Bulk vs. single cell}



\section{Research goals}
\label{sec:motivation}
\todo{Write research goals}



