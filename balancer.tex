\chapter{Effects of \acsp{sv} on gene expression and chromatin organisation}
\label{sec:balancer}

\section{Motivation}

\section{Background}
\subsection{Impact of \texorpdfstring{\acp{sv}}{SVs} in health and disease}
\subsection{Three-dimensional chromatin organization and recent findings on its role in gene regulation}


\section{Design of the study}

\explain{\cyo}{chromosome 2 balancer, derived from original work of
\citet{Oster1956}, which carries the \textit{Cy}
allele, hence showing a characteristic phenotype of curly wings in adult flies
\marginfig{dm_cyo.pdf}}

%\margintext{\marginfig{balancer_chroms.pdf}\captionof{figure}{Lallalala}}

\figuretwocolumns[0.38]
    {balancer_chroms.pdf}{balancer_chroms}{Genome overview}{Schematic of the
        four chromosomes of (female) \textit{D. melanogaster} flies of the wild
        type line $F_0$ and of the double balancer cross $F_1$. \Ac{cyo} red,
        \ac{tm3} blue. }
    {balancer_crossing_scheme.pdf}{balancer_crossing_scheme}{Crossing scheme}{
        The fly lines used in this study are derived from a homozygous wild type
        line, denoted by \textit{+/+;+/+} or $F_0$ and a double balancer line
        (\textit{w;If/CyO;Sb/TM3,Ser}). After a first cross adult flies are
        selected for markers of both balancer chromosomes, yielding an $F_1$
        generation (a.k.a. double balancer cross). A backcross with the initial
        wild type line generates a pool of four different genotypes ($N_1$)
        of which on average 25\% of both chromosome 2 and 3 are balancer
        chromosomes.}

\begin{itemize}
\item Balancer chromosomes: \cite{Muller1928,Oster1956,Tinderholt1960,Miller2016}
\end{itemize}




\section{Mutational landscape of balanced chromosomes}

\acreset{snv}
\subsection{\Acp{snv}}
I used whole genome sequencing data of both $F_0$ and $F_1$ cross
simulataneously to call \acp{snv}. The comparison of genotypes of the homozygous
wild type line and the heterozygous cross enabled me to assign mutations to
individual haplotypes: if, for example, a \snv was called heterozygous in the
cross and homozygous alternative in the wild type sample, the variant is located
on the wild type chromosome and not on the homologous balancer chromosome.

Of the 761,348 detected \acp{snv} on \ac{chr2} and \ac{chr3}, 38.9\% can be
assigned to the balancer chromosomes, 29.5\% to the wild type chromosomes and
29.8\% are shared between both. Only a fraction of 1.8\% of \acp{snv} did not
fit the expected genotypes, which can mostly be attributed to regions in the
wild type chromosomes not being fully homozygous.
On average every 210bp there is a \snv distinguishing balancer and wild type
haplotypes.

\begin{table}[ht]
    \centering
        %     wild type-specific    224,353
        %     balancer-specific     296,168
        %     common                226,907
        %     -----------------------------
        %chr2 wild type-specific 109654     48800648 2246.978360
        %chr2         errorneous    857     48800648   17.561242
        %chr2       heterozygous   4814     48800648   98.646231
        %chr2  balancer-specific 132594     48800648 2717.054085
        %chr2             common 114128     48800648 2338.657470
        %chr3         errorneous   1003     60189558   16.664020
        %chr3  balancer-specific 163574     60189558 2717.647470
        %chr3       heterozygous   7246     60189558  120.386330
        %chr3             common 112779     60189558 1873.730324
        %chr3 wild type-specific 114699     60189558 1905.629545
        %chrX         errorneous    134     23542271    5.691889
        %chrX       heterozygous   4148     23542271  176.193707
        %chrX  balancer-specific  22513     23542271  956.279876
        %chrX             common  52082     23542271 2212.275952
        %chrX wild type-specific  19870     23542271  844.013732
    \begin{tabu}{lrrr}
        \toprule
        Genotype & \ac{chr2} & \ac{chr3} & \ac{chrX} \\
        \midrule
        balancer-specific  & 2717.1 & 2717.6 &  956.3 \\
        wild type-specific & 2247.0 & 1905.6 &  844.0 \\
        shared             & 2338.7 & 1873.7 & 2212.3 \\
        wt heterozygous    &   98.6 &  120.4 &  176.2 \\
        errorneous         &   17.6 &   16.7 &    5.7 \\
        \bottomrule
    \end{tabu}
    \tabcap{snvs2}{Number of \acp{snv} per Megabase}{The
        majority of \snv calls are in concordance with the study design
        (balancer-specific, wild type-specific, or shared), except in short
        regions of remaining heterozygosity of the wild type chromosomes and
        a negligible number of false calls.
        Note that \ac{chrX} is not balanced but also paired with a homologue
        from a divergent line.}
\end{table}

The number of balancer-specific \acp{snv} is around 1.3 times higher than wild
type-specific ones. Moreover, when I compared our fly lines to the Drosophila
reference panel (DGRP) \citep{Mackay2012,Huang2014}, a panel of fully sequenced
inbred \species{D. melanogaster} lines derived from a natural population, I
found that a striking majority of \acp{snv} is represented in the panel, yet
balancer-specific \acp{snv} to a lower fraction (86.1\%) than wild type-specific
ones (92.1\%). These observations are not surprising since balancer chromosomes,
due to their inability to undergo recombination, accumulate and likely tolerate
more mutations over time than normal chromosomes.

I was also interested in the distribution of base substitutions, i.e.
the \explain{mutation spectrum}{as a summary of the total set of
mutations I count the relative frequencies of different base substitutions
(e.g. \texttt{C>T}) and their trinucleotide contexts (one base up- and
downstream). The spectrum of somatic \acp{snv} is often analysed in cancer
genomics in this way, in order to extract \emph{mutational signatures} that are
linked to mutagenic processes}. The idea behind this is that
balancer chromosomes had been derived by X-ray mutagenesis and that this
could have led to a different distribution of base substitutions. However, after
removing \acp{snv} shared with the DGRP we did not observe any striking
differences in the \snv signatures between wild type and balancers (see
\cref{fig:signatures}).



\subsection{\texorpdfstring{\Aclp{cnv}}{Copy number variants}}
\label{sec:balancer_cnv}

\figuretwocolumns
    {SVsize_DEL.pdf}{svsize_del}{Deletion size distribution}
    {The size distribution of final deletion calls (based on \delly and
    \freebayes) is shown. Common deletions (present in both balancer and wild
    type chromosomes) were excluded in the from 160~bp on. A higher number of
     balancer- over wild type-specific deletions was observed and this ratio
     increases with larger sizes.}
    {SVsize_DUP.pdf}{svsize_dup}{Duplication size distribution}
    {The final duplication set consisting of manually validated tandem
    duplication and a set of three non-tandem duplications is shown.}


\paragraph{Deletions}

\begin{table}[ht]
    \centering
    \begin{tabu}{lrrrr}
        \toprule
        Deletion subset           & \#Calls &  V &  I & Emp. \fdr\\
        \midrule
        \emph{<50~bp}             &   3,072 &  0 &  0 & \emph{unknown} \\
        50 - 159~bp               &     737 & 24 &  1 &     4\% \\
        160+~bp, low mappability  &      75 & 25 &  0 &     0\% \\
        160+~bp, high mappability &     395 & 24 &  1 &     4\% \\
        \midrule
        Weighted average          &   4,279 &    &    & 0.375\% \\
        \bottomrule
    \end{tabu}
    \tabcap{pcrresults}{\Ac{pcr} for deletion validation}{Number of calls as
        well as number number of validated (\textbf{V}) and invalidated
        (\textbf{I}) loci as determined by \pcr. The proportion of validated
        calls determines the empirical \ac{fdr} (\textbf{Emp. \fdr}).
        Calls below 160~bp were not \pcr-validated.}
\end{table}


\paragraph{Duplications}

\figuretextwidth{dup_validation_example.pdf}{dup_validation} % DUP00003911
    {Signals used for duplication validation shown in an example}
    {A tandem duplication at locus \textit{chr2R:7,520,066−7,526,996} is shown,
    which was predicted based on paired-end read signature using \delly.
    The additional signals
    included for a visual validation are (1) a mappability track (fraction
    of uniquely mappable reads), (2) the bi-allelic frequency (fraction of
    reads supporting the alternative allele of an \snv), and (3)
    the total read coverage (as genomic coverage, in 100bp windows). Bi-allelic
    frequency is typically the clearest signal to discriminate true from false
    copy number variants, but it reuqires the presence of \acp{snv}. Note }
    \todo{describe bi-allelic frequency in intro; or here, if not there.}


There are further examples in the supplementary material (see \crefrange{fig:dup_validation_wt}{fig:dup_validation_false}).

\paragraph{Summary}



\section{xxx}
Karyotype\footnote{
    Described in terms of cytological bands, i.e. \cyo = \textit{2Lt-22D1
    33F5-30F 50D1-58A4 42A2-34A1 22D2-30E 50C10-42A3 58B1-2Rt} and
    \ac{tm3} = \textit{3Lt-65E 85E-79E 100C-100F2 92D1-85E 65E-71C
    94D-93A 76C-71C 94F-100C 79E-76C 93A-92E1 100F3-3Rt}.
    Source: \url{https://bdsc.indiana.edu/stocks/balancers/balancer_bps.html}.}

\figuretextplusmargin{balancer_hiC_rearrangements.png}{balancer_hiC_rearrangements}
    {Major rearrangements of the balancer chromosomes revealed by Hi-C}{
    Haplotype-resolved Hi-C contact frequency maps of chromosomes 2 and 3 in
    respect to the wild type reference genome are shown (A). Details of the data
    processing can be found in \cref{sec:suppl_hic}. The bottom triangles show
    characteristic ``bowtie-shaped'' patterns as well as gaps on the diagonal
    that demarcate breakpoint junctions of the rearranged balancer chromosomes.
    The respective sections of the reference genome are color-coded, with a grey
    circle representing centromers.
    These junctions can be followed to reconstruct the relative order of the
    balancer chromosomes such as shown in the top panel (B). Vertical ``bowties''
    represent a connection of segments in same orientation where as horizontal
    ones mark junction of segments that are inverted in respect to one another.
    The fully reconstructed order of the balancer genomes is shown in the bottom
    panel (C). This matches previous annotation from karyotyping\footnotemark.
    }
