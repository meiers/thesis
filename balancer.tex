\chapter{Effects of \acsp{sv} on gene expression and chromatin organisation}

The research projects described in previous chapters focused on the technical 
aspects of identifying and characterising \acp{sv}. In the project presented 
now I teamed up with Yad Ghavi-Helm, Aleksander Jankowski and Eileen Furlong
to study the functional impact of \acp{sv} in general and of large chromosomal
rearrangements in particular.

\section{Motivation}

\Ac{sv} have been linked to functional ...\todo{Write a motivaiton including
(1) general effects of SVs (2) enhancer hijacking (3) studies biased towards
extreme phenotypes (4) balancer chromosomes}

We use \textit{Drosophila melanogaster}.

\section{Study design}

We use a fly line containing \explain{balancer chromosomes}{artificially mutated
chromosomes carrying recessive deleterious mutations. Homozygous lethal
mutations can be kept in a ``normal'' chromosome using balancer chromosomes
since only heterozygous offspring survives (a balanced population, hence the
name)}, which are typically used to prevent recombination in fly genetics and
which are also known to be highly rearranged compared to their wild type
homologues. Our fly line is balanced on both major chromosomes (\textsc{chr2}
and \textsc{chr3}).

...

Balancer chromosomes carry genetic markers such as \explain{CyO}{name of a chromosome 2 balancer which creates a characteristic phenotype of curly wings in adult flies\marginfig{dm_cyo.pdf}} and TM3, which are visible in the adult stage.