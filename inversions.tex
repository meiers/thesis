\chapter{Complex inversions in the human genome}
\label{sec:inversions}

See suppl. info in \cref{sec:suppl_inversions}.

In the motivation to this chapter I mention Adrian's \explain{Four-primer
strategy}{\marginfig{four_primer_strategy.pdf} Four separate PCRs are run using
primer pairs FR, fr, Ff and Rr.
In samples not carrying the inversion only FR and fr will yield bands, in
homozygous carriers only Ff and Rr, and in heterozygous carriers all four
reactions will yield bands.} and I explain it quickly in the margin.


Papers:
\begin{itemize}
\item my co-authorship: \citep{Sudmant2015}
\item HGP and Celera: \citep{Lander2001,Venter2001}
\item HapMap: \citep{InternationalHapMapConsortium2005,Frazer2007,Altshuler2010}
\item 1000G: \citep{Durbin2010,1000GenomesProjectConsortium2012,Auton2015}
\item PacBio papers: \citep{Chaisson2014,Pendleton2015}
\item Tools: \blasr, \last, \mummer, \bwamem, \quiver
\end{itemize}



% QUOTE FROM Sudmant et al, 2015:
% To further explore inversion sequence complexity, we performed a battery of
% targeted analyses, leveraging PacBio resequencing of fos- mids (targeting 34
% loci), sequencing by Oxford Nanopore Minion (60 loci) and PacBio (206 loci)
% of long-range PCR amplicons, and data for 13 loci from another sample (CHM1)
% sequenced by high- coverage PacBio WGS14. Altogether we verified and further
% characterized 229 inversion sites, 208 using long-read data and 21 by PCR
% (Supplementary Table 15), increasing the number of known validated inversions
% by .2.5-fold.
%
% SUMMARY OF SUPPL. TABLE 15
% sheet A. (sheet B contains 21 additional (!) PCR-validated loci)
%
%           pacbio
% validation   N   Y
%   negative   7  36
%   positive  38 170
%
%           minion
% validation   N   Y
%   negative  37   6
%   positive 154  54
%
% class
% Complex    Inv and Del   Inverted Dup MultiDel with Inv    Simple Inv
%      11             28            113                14            42

\begin{table}[ht]
    \begin{adjustbox}{inner,minipage=[b]{\textplusmargin}}
            \begin{minipage}[b]{0.54\textplusmargin}
                \centering
                \begin{adjustbox}{tabular=lrrr,valign=b}
                    \toprule
                    Method      & \#Total & \#Val. & \#Inval. \\
                    \midrule
                    \pcr        & 96  & 21  &   - \\
                    PacBio      & ??? & ??? & ??? \\
                    ONT MinION  & 60  &  35 &   2 \\
                    \bottomrule
                \end{adjustbox}
            \end{minipage}
            \hspace{0.04\textplusmargin}
            \begin{minipage}[b]{0.42\textplusmargin}
                \centering
                \begin{adjustbox}{tabular=lr,valign=b}
                    \toprule
                    Class & Count \\
                    \midrule
                    Simple inversion     &  42 \\
                    Inverted duplication & 110 \\
                    Inv. and del.        &  29 \\
                    Inv. and multi-del.  &  14 \\
                    Highly complex       &  11 \\
                    \bottomrule
                \end{adjustbox}
            \end{minipage}\\[-\baselineskip]%
            \begin{minipage}[t]{0.54\textplusmargin}
                \centering
                \tabcap{inv_validations}{Inversion validations}{Number of
                predicted inversions analysed (\#Total), validated (\#Val.) and
                invalidated (\#Inval.) using one of several techniques.
                Not included are orthorgonal validation methods performed by
                collaborators.}
            \end{minipage}
            \hspace{0.04\textplusmargin}
            \begin{minipage}[t]{0.42\textplusmargin}
                \centering
                \tabcap{inv_classes}{Inversion classes}{Summary of classes of
                    inversions found at the 206 resolved loci, which are also
                    illustrated in \cref{fig:complex_invs} (\emph{Inv. and del.}:
                    Inversion flanked by deletion, \emph{Inv. and multi-del.}:
                    Inversion flanked by multiple deletions)}
            \end{minipage}
        \end{adjustbox}
\end{table}





\figuretextwidth{complex_invs.pdf}
    {complex_invs}
    {Complex inversion types revealed by PacBio and ONT MinION reads}
    {Examplatory loci were selected to represent the most most commonly seen
    classes of complex inverted variants (listed in table XXX).
    Green lines denote inversions predicted by \delly. PacBio circular
    consesus sequences and raw ONT MinION reads are compared to their reference
    locus via dotplots using maximal unique matches (REF) of
    lengths at least 10 (PacBio) or 8 (MinION). Axes denote sequence lengths in
    base pairs. Original read names are included for later reference.}

\figuretextwidth{complex_invs_artifacts.pdf}
    {complex_invs_artifacts}
    {Loci showing three alleles, likely resulting from PCR artifacts}
    {Example locus showing more than two alleles in a single diploid sample,
    plotted the same way as in \cref{fig:complex_invs}.}



\figuretwocolumns{inv_assemblies_qualities.pdf}
    {inv_assemblies_qualities}
    {Quality of assembled loci}
    {Overview of length and quality of the 210 loci (listed in
    \cref{tab:inversionlocilist}) that were assessed using
    any of three technologies. Quality is approximated by number of mismatches
    to the reference genome, taking into account known variantion of the sample.
    Note that unlike Illumina-assembled reads, PacBio assemblies are restricted
    to the size of the amplicons.}
        {inv_assemblies_sizes.pdf}
    {inv_assemblies_sizes}
    {Sizes of rearranged sequences}
    {Sizes were determined from assembled loci using \acs{last}-split as
    described in section XXX. Note that the overall size of the amplicons
    is limited, which potentially explains shorter sizes for the inverted part
    in loci with higher complexity.}

\figuretwocolumns{inv_assemblies_del_sizes.pdf}
    {inv_assemblies_del_sizes}
    {Size of flanking deletions}
    {Size of deletions (detected via \acs{last}-split, see section XXX) next to
    inverted sequences in all assembled loci that were classified as
    ``simple inversion'', ``inversion flanked by deletion'', or ``inversion
    flanked by multiple deletions''.}
        {inv_assemblies_microhomology.pdf}
    {inv_assembles_microhomology}
    {Breakpoint characteristics of complex loci vs. deletions}
    {Micro-deletions or -duplications (a.k.a. micro-homology) as identified by
    \last-split are reported for assembled loci with exactly two breakpoints
    (199/210) as well as for a set of 12,206 single-breakpoint deletions (REF needed).
    TODO: information on tests performed}

