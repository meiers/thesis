\chapter{Complex inversions in the human genome}
\label{sec:inversions}

See suppl. info in \cref{sec:suppl_inversions}.
\todo{write chapter}

\figuretextwidth{complex_invs.pdf}
    {complex_invs}
    {Complex inversion types revealed by PacBio and ONT MinION reads}
    {Examplatory loci were selected to represent the most most commonly seen
    classes of complex inverted variants as listed in table XXX. PacBio circular
    consesus sequences and raw ONT MinION reads are compared to their reference
    locus via dotplots using maximal unique matches (MUMs) from Mummer3 (REF) of
    lengths at least 10 (PacBio) or 8 (MinION). Axes denote sequence lengths in
    base pairs.}

\figuretextwidth{complex_invs_artifacts.pdf}
    {complex_invs_artifacts}
    {Loci showing three alleles, likely resulting from PCR artifacts}
    {Example locus showing more than two alleles in a single diploid sample,
    plotted the same way as in \cref{fig:complex_invs}.}
