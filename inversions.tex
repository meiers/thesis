\chapter{Complex inversions in the human genome}
\label{sec:complex_invs}


In 2014 and 2015 I was invited to collaborate with a large consortium of
scientists on the 1000 Genomes Project. My supervisor Jan Korbel was the
co-leader of the structural variation subgroup and together with my colleagues
\tobias, \adrian, \benjamin, \markus, and \andreas
I was involved in the validation and characterization of inversions.
This chapter covers my work for the 1000 Genomes Project, which not only turned
out to solve an interesting mystery but also resulted in a co-authorship on
\cite{Sudmant2015}. I continue by describing subsequent work, including a side
project on sequence match visualization that came into being from collaboration
with \markus (\cref{sec:maze}), as well as an analysis of inversion breakpoints
(\cref{sec:breakpoints}). These results have never been published in a journal,
but were presented partially in form of a poster at the German Conference for
Bioinformatics 2016 in Berlin. There is supplementary information to this
chapter enclosed in the appendix (\cref{sec:suppl_inversions}).



\section{Motivation}
\subsection{The 1000 Genomes Project}
\label{sec:1000G}

The 1000 Genomes Project was the effort of a large international panel of
scientists to capture the genetic diversity of the human population as
comprehensively as possible. The project can be regarded as the hitherto final
step in a series of projects that unraveled and characterize the human genome:
The Human Genome Project, which, founded in 1990, was the first large-scale
collaborative work in biology, as well as their private competitor Celera
Genomics (from 1998) aimed at completing the first human genome sequence.
These tremendous efforts, published simultaneously in early
2001\citep{Lander2001,Venter2001}, had both used DNA of a few individuals (among
them Celera’s then-CEO Craig Venter) to determine an average human genome. The
focus was shifted to the variation present in the human population with the
initiation of the International HapMap Project in 2002. By performing
whole-genome \snv genotyping on hundreds of individuals from multiple continents,
including family-offspring trios, the project generated a first haplotype map of
the human population\citep{InternationalHapMapConsortium2005,Frazer2007,Altshuler2010}.
Over the years the HapMap Project was scaled up to more than 1,000 individuals,
to discover rare \acp{snv}  (minor allele frequency < 5\%) and to call copy number
variants. The samples originally collected for the HapMap Project, which are
still available today as immortalized lymphoblastoid cell lines, were also used
by the 1000 Genomoes Project. Starting long before my own participation, with
the pilot phase publication dating back to 2010\citep{Durbin2010}, the project
even extended this set of samples by multiple other
populations\citep{1000GenomesProjectConsortium2012}. However, the fundamental
novelty was to apply low-coverage whole-genome sequencing which allows to assess
a much broader spectrum of variants in a much larger portion of the genome.

In the third and final phase the project had performed mean 7.4x-coverage
sequencing on 2,504 samples and additionally expanded the palette of
computational methods that were applied as well as added new types of variation
to be studied\citep{Auton2015}. The final data set consists of approximately
88 million phased variants, which was estimated to include more than 99\% of
non-rare \acp{snv} in the studied populations. Besides novel insides on, for example,
seemingly dispensable genes, the primary achievement of the project was to make
an unprecedented resource of genetic variation available to the public. This
came along with technical advances in methods and standardizations such as the
nowadays commonly used file format \vcf\footnote{Established for the
1000 Genomes Project, the format is nowadays maintained by the Global Alliance
for Genomics \& Health (\url{https://www.ga4gh.org}}.
The 1000 Genomes data has since been utilized in many research studies, for
example to refine human population history\citep{Veeramah2014}, to
impute\citep{Howie2012} missing \acp{sv} in many genome-wide association studies
(e.g. \cite{Wood2014}), or to discriminate somatic from germline mutation in
cancer studies\citep{Hiltemann2015}. At last, since the start of the 1000
Genomes Project a considerable number of similarly large sequencing studies has
popped up that focus on specific populations\citep{UK10K,Sulem2015,Telenti2016}
or disease\citep{Campbell2017} and which eventually scale up the catalogue of
known variation in the human population.





\subsection{Predicted inversions and the validation problem}

The Structural Variation subgroup of the 1000 Genomes Project focused on the
identification of various classes of genomic rearrangements. With a total of 83
scientists involved and led by eight principle investigators with different
expertise, the group assembled the to date most comprehensive resource of
structural variation in the human population. Copy number variants, inversions,
mobile element insertions and other types of \sv were predicted from
low-coverage whole-genome sequencing data, statistically phased and
validated using a number of different approaches.

Among his many contributions to this project, \tobias had predicted inversions
using the paired-end signature-based \sv detection tool \delly. This set of
inversions, which was strictly filtered according to validation results and
population genetics aspects, contains a total of 786 inversion calls in the
range of 500~bp~-~50~kb (median size 1.70~kb) and ended up being the only
inversion call set used in the project. In comparison to phase one of the 1000
Genomes Project, this is a more than 15-fold increase in the number of
inversions.

Before its final release, this call set (just like any other call set in the
project) had to undergo cycles of validation and filtering. \adrian and
\benjamin were in charge of running \pcr-based validation experiments of these
inversions. In a nutshell they applied a \explain{Four-primer strategy}
{\marginfig{four_primer_strategy.pdf} Four separate PCRs
are run using primer pairs FR, fr, Ff and Rr. In samples
not carrying the inversion only FR and fr will yield bands,
in homozygous carriers only Ff and Rr, and in heterozygous
carriers all four reactions will yield bands.}
on inversion loci, which is capable of distinguishing the genotype of inversions
based on the combination of bands resulting from the \acp{pcr}. However,
this requires that the breakpoints of the inversion were accurately predicted
and also lie in regions accessible to targeted \pcr amplification. The latter
was initially thought to be a problem, as inversions are known to often be
flanked by inverted repetitive sequence due to their origination through \nahr
(see \cref{sec:mechanisms}\todo{explain NAHR in this section!}).

When they tested \pcr verification experiments on a subset of 96 predicted
inversion loci they were puzzled: for the vast majority of loci the received
band patterns did not match the expectations, neither clearly validating nor
invalidating the locus. When counted as invalidations, these results would have
given rise to an \fdr of up to 80\%. Despite multiple trials to improve \pcr
conditions and primer locations, the experimental verification of the inversion
call set remained unsuccessful for a long time and posed a scientific mystery.






\section{Long-read sequencing of inversion loci unravels unexpected levels of complexity}


% QUOTE FROM Sudmant et al, 2015:
% To further explore inversion sequence complexity, we performed a battery of
% targeted analyses, leveraging PacBio resequencing of fos- mids (targeting 34
% loci), sequencing by Oxford Nanopore Minion (60 loci) and PacBio (206 loci)
% of long-range PCR amplicons, and data for 13 loci from another sample (CHM1)
% sequenced by high- coverage PacBio WGS14. Altogether we verified and further
% characterized 229 inversion sites, 208 using long-read data and 21 by PCR
% (Supplementary Table 15), increasing the number of known validated inversions
% by .2.5-fold.
%
% SUMMARY OF SUPPL. TABLE 15
% sheet A. (sheet B contains 21 additional (!) PCR-validated loci)
%
%           pacbio
% validation   N   Y
%   negative   7  36
%   positive  38 170
%
%           minion
% validation   N   Y
%   negative  37   6
%   positive 154  54
%
% class
% Complex    Inv and Del   Inverted Dup MultiDel with Inv    Simple Inv
%      11             28            113                14            42


\begin{table}[ht]
    \begin{adjustbox}{inner,minipage=[b]{\textplusmargin}}
            \begin{minipage}[b]{0.54\textplusmargin}
                \centering
                \adjustbox{valign=b,padding=0ex,margin=0ex}{
                \begin{tabu}{Xrrr}
                    \toprule
                    Method      & \#Total & \#Val. & \#Inval. \\
                    \midrule
                    \pcr        &  96 &  27 &   2 \\
                    PacBio      & 308 & 186 &  37 \\
                    ONT MinION  &  96 &  54 &   6 \\
                    Fosmid      &  34 &  28 &   4 \\
                    \midrule
                    Total       & 308 & 229 &  47 \\
                    \bottomrule
                \end{tabu}
                }
            \end{minipage}
            \hspace{0.04\textplusmargin}
            \begin{minipage}[b]{0.42\textplusmargin}
                \centering
                %\begin{adjustbox}{tabular=lr,valign=b}
                \adjustbox{valign=b,padding=0ex,margin=0ex}{
                \begin{tabu}{Xr}
                    \toprule
                    Class & Count \\
                    \midrule
                    Simple inversion     &  42 \\
                    Inverted duplication & 110 \\
                    Inv. and del.        &  29 \\
                    Inv. and multi-del.  &  14 \\
                    Highly complex       &  11 \\
                    \bottomrule
                \end{tabu}
                }
                %\end{adjustbox}
            \end{minipage}\\[-\baselineskip]%
            \begin{minipage}[t]{0.54\textplusmargin}
                \centering
                \tabcap{inv_validations}{Inversion validations}{Number of
                predicted inversions analysed (\#Total), validated (\#Val.) and
                invalidated (\#Inval.) using one of several techniques performed
                at EMBL or by collaborators.}
            \end{minipage}
            \hspace{0.04\textplusmargin}
            \begin{minipage}[t]{0.42\textplusmargin}
                \centering
                \tabcap{inv_classes}{Inversion classes}{Summary of classes of
                    inversions found at 206 long read-resolved loci
                    (\emph{Inv. and del.}:
                    Inversion flanked by deletion, \emph{Inv. and multi-del.}:
                    Inversion flanked by multiple deletions)}
            \end{minipage}
        \end{adjustbox}
\end{table}
\todo{fix left-shifting caused (most likely) by adjustbox}





\figuretextwidth{complex_invs.pdf}
    {complex_invs}
    {Complex inversion types revealed by PacBio and ONT MinION reads}
    {Examplatory loci were selected to represent the most commonly seen
    classes of complex inverted variants (listed in \cref{tab:inv_classes}).
    Green lines denote inversions predicted by \delly. PacBio circular
    consesus sequences and raw ONT MinION reads (vertical) are compared to their
    reference locus (horizontal) via dotplots using maximal unique matches
    of lengths at least 10 (PacBio) or 8 (MinION). Axes denote sequence lengths
    in base pairs. Original read names are included for later reference. See
    \cref{sec:dotplot} for an explanation of the concept of dotplots.}


\subsection{Artifacts in amplicon sequencing}
\label{sec:complex_invs_artifacts}


\figuretextwidth{complex_invs_artifacts.pdf}
    {complex_invs_artifacts}
    {Loci showing three alleles, likely resulting from PCR artifacts}
    {Example locus showing more than two alleles in a single diploid sample,
    plotted the same way as in \cref{fig:complex_invs}.}



\section{Assembly of PacBio reads yields sequence information at basepair resolution}
\label{sec:breakpoints}




\figuretwocolumns{inv_assemblies_qualities.pdf}
    {inv_assemblies_qualities}
    {Quality of assembled loci}
    {Overview of length and quality of the 210 loci (listed in
    \cref{tab:inversionlocilist}) that were assessed using
    any of three technologies. Quality is approximated by number of mismatches
    to the reference genome, taking into account known variantion of the sample.
    Note that unlike Illumina-assembled reads, PacBio assemblies are restricted
    to the size of the amplicons.}
        {inv_assemblies_sizes.pdf}
    {inv_assemblies_sizes}
    {Sizes of rearranged sequences}
    {Sizes were determined from assembled loci using \acs{last}-split as
    described in section XXX. Note that the overall size of the amplicons is
    limited, which potentially explains shorter sizes for the rearranged part in
    loci with higher complexity.}

\figuretwocolumns{inv_assemblies_del_sizes.pdf}
    {inv_assemblies_del_sizes}
    {Size of flanking deletions}
    {Size of deletions (detected via \acs{last}-split, see section XXX) next to
    inverted sequences in all assembled loci that were classified as
    ``simple inversion'', ``inversion flanked by deletion'', or ``inversion
    flanked by multiple deletions''.}
        {inv_assemblies_microhomology.pdf}
    {inv_assembles_microhomology}
    {Breakpoint characteristics of complex loci vs. deletions}
    {Micro-deletions or -duplications (a.k.a. micro-homology) as identified by
    \last-split are reported for assembled loci with exactly two breakpoints
    (199/210) as well as for a set of 12,206 single-breakpoint deletions (REF needed).
    TODO: information on tests performed}



\section{Maze: A tool for match visualization and breakpoint inpsection}
\label{sec:maze}


\subsection{Interactive match visualization in the web browser}
\label{sec:dotplot}


\figuretextwidth{maze_screenshot.pdf}{maze}
    {Screenshot from \maze}
    {On the left, \maze displays a dotplot of two sequences (here inversion
    assembly, vertically, vs. its reference locus, horizontally) and highlights
    sequence alignments (red). On the right, the sequence alignment around the
    breakpoints is highlighted and classified into cases of micro-insertions,
    blunt ends or micro-homology.}



...\acs{mummer}'s dotplot function\footnote{More precisely, by the \texttt{mummerplot} command.
Examples can be found online at \url{http://mummer.sourceforge.net/examples/}.}.



\subsection{Breakpoint identification and characterization}



\section{Conclusions}









Papers:
\begin{itemize}
\item my co-authorship: \citep{Sudmant2015}
\item HGP and Celera: \citep{Lander2001,Venter2001}
\item HapMap: \citep{InternationalHapMapConsortium2005,Frazer2007,Altshuler2010}
\item 1000G: \citep{Durbin2010,1000GenomesProjectConsortium2012,Auton2015}
\item PacBio papers: \citep{Chaisson2014,Pendleton2015}
\item Papers finding complex inversions: Talkowski paper (2017)
    \url{https://www.ncbi.nlm.nih.gov/pubmed/28260531}, where they use jumping
    libraries to find even more complex classes. Ashley's Strand-seq inversion
    paper which at least refers to complex inversions
    \url{https://www.ncbi.nlm.nih.gov/pmc/articles/PMC5088599/}. Another study
    (\url{https://www.ncbi.nlm.nih.gov/pmc/articles/PMC4490614/}) found exactly
    one complex inversion, shown in figure 7
\item On the

\item Tools: \blasr, \last, \mummer, \bwamem, \quiver, \celeraassembler, \maze
\item Dotplots: \citep{Fitch1969,Gibbs1970}
\end{itemize}
